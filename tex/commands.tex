% -----------------------------------------------------------------------------
% Pacotes fundamentais
% -----------------------------------------------------------------------------
\usepackage{xcolor}
\newcommand\myworries[1]{\textcolor{red}{[#1]}}

% Escolhendo a fonte
% Consultar catálogo de fontes: http://www.tug.dk/FontCatalogue/
% \usepackage{lmodern}        % Usa a fonte Latin Modern (Serifada, tipo Times New Roman
\usepackage{fourier}        % Usa a fonte Utopia Regular with Fourier
% \usepackage[T1]{fontenc}    % ^^^^

% \usepackage{helvet}         % Usa a fonte Helvetica (Tipo Arial)		
% \renewcommand{\familydefault}{\sfdefault}   % tira o serifado

\usepackage[T1]{fontenc}		% Selecao de codigos de fonte.
\usepackage[utf8]{inputenc}		% Codificacao do documento (conversão automática dos acentos)
\usepackage{indentfirst}		% Indenta o primeiro parágrafo de cada seção.
\usepackage{color}				% Controle das cores
\usepackage{tikz}				% Inclusão de gráficos
\usepackage{graphicx}			% Inclusão de gráficos
\usepackage{microtype} 			% para melhorias de justificação
% -----------------------------------------------------------------------------
% Pacotes adicionais, usados no anexo do modelo de folha de identificação
% -----------------------------------------------------------------------------
\usepackage{multicol}
\usepackage{multirow}
% -----------------------------------------------------------------------------
% Pacotes adicionais, usados apenas no âmbito do Modelo Canônico do abnteX2
% -----------------------------------------------------------------------------
\usepackage{lipsum}				% para geração de dummy text
% -----------------------------------------------------------------------------
% Pacotes de citações
% -----------------------------------------------------------------------------
\usepackage[brazilian,hyperpageref]{backref}	 % Paginas com as citações na bibliografia
\usepackage[alf,abnt-etal-list=3,abnt-etal-cite=3, abnt-emphasize=bf]{abntex2cite}	% Citações padrão ABNT
\usepackage{pdflscape}
\usepackage{footnote}
\usepackage{pdfpages}
\usepackage[skip=0pt]{caption} % Configurado para não ter espaço extra após o rótulo da figura
% \setlength{\abovecaptionskip}{0pt plus 0pt minus 0pt} % https://tex.stackexchange.com/questions/45990/how-can-i-modify-vertical-space-between-figure-and-caption
% \setlength{\belowcaptionskip}{0pt plus 0pt minus 0pt}
% \usepackage{caption}

% -----------------------------------------------------------------------------
% Pacotes adicionados por @leolleocomp
% ----------------------------------------------------------------------------- 
\usepackage{booktabs}
\usepackage{adjustbox}
\usepackage{subcaption}
\usepackage[labelfont=bf]{caption}
\usepackage{gensymb}
\usepackage{amsmath}
\usepackage{array}
\usepackage{float}
\usepackage{xcolor,colortbl}
\usepackage{longtable}
\usepackage{scalefnt}
\usepackage{listings}			% inserir codigo fonte


% -----------------------------------------------------------------------------
% Pacotes adicionados por @Gabrielr2508
% ----------------------------------------------------------------------------- 
\usepackage{hyperref}

% Para forçar quebra de linhas em urls muito longos
% https://tex.stackexchange.com/questions/3033/forcing-linebreaks-in-url
% \RequirePackage[hyphens]{url}
% \PassOptionsToPackage{hyphens}{url}\usepackage{hyperref}

% \usepackage{hyperref}

\usepackage{tocloft}
% -- permite a adição de células especiais em tabelas
\newcommand{\specialcell}[2][c]{%
  \begin{tabular}[#1]{@{}c@{}}#2\end{tabular}}

\newcounter{equationset}
\newcommand{\equationset}[1]{% \equationset{<caption>}
  \refstepcounter{equationset}% Step counter
  \noindent\makebox[\linewidth]{Equação ~\theequationset: #1}
 }

%--------------------------------------------------------------------------------
% Adequação dos títulos dos capitulos, seções, subseções às normas da Univasf
% Added by @Gabrielr2508
%--------------------------------------------------------------------------------
\renewcommand{\ABNTEXchapterfont}{\fontseries{b}}
\renewcommand{\ABNTEXchapterfontsize}{\normalsize}

\renewcommand{\ABNTEXsectionfont}{\fontseries{m}}
\renewcommand{\ABNTEXsectionfontsize}{\normalsize}

\renewcommand{\ABNTEXsubsectionfont}{\fontseries{b}}
\renewcommand{\ABNTEXsubsectionfontsize}{\normalsize}

\renewcommand{\ABNTEXsubsubsectionfont}{\fontseries{m}}
\renewcommand{\ABNTEXsubsubsectionfontsize}{\normalsize}

%--------------------------------------------------------------------------------
% CONFIGURAÇÕES DE PACOTES
% Configurações do pacote backref
%--------------------------------------------------------------------------------
% Usado sem a opção hyperpageref de backref
\renewcommand{\backrefpagesname}{Citado na(s) página(s):~}
% Texto padrão antes do número das páginas
\renewcommand{\backref}{}
% Define os textos da citação
\renewcommand*{\backrefalt}[4]{
	\ifcase #1 %
		%Nenhuma citação no texto.%
	\or
		Citado na página #2.%
	\else
		Citado #1 vezes nas páginas #2.%
	\fi}%

%--------------------------------------------------------------------------------
% Configurações de aparência do PDF final
%--------------------------------------------------------------------------------
% alterando o aspecto da cor azul
\definecolor{blue}{RGB}{41,5,195}

% informações do PDF
\makeatletter
\hypersetup{
     	%pagebackref=true,
		pdftitle={\@title},
		pdfauthor={\@author},
    	pdfsubject={\imprimirpreambulo},
	    pdfcreator={LaTeX with abnTeX2},
		pdfkeywords={abnt}{latex}{abntex}{abntex2}{relatório técnico},
		colorlinks=true,			% false: boxed links; true: colored links
    	linkcolor=black,				% color of internal links
    	citecolor=black,				% color of links to bibliography
    	filecolor=black,			% color of file links
		urlcolor=black,
		bookmarksdepth=4
}
\makeatother
% ---

% ---
% Espaçamentos entre linhas e parágrafos
% ---

% O tamanho do parágrafo é dado por:
\setlength{\parindent}{1.3cm}

% Controle do espaçamento entre um parágrafo e outro:
\setlength{\parskip}{0.2cm}  % tente também \onelineskip

% Controle do espaçamento em listas de itens
\setlist[itemize]{leftmargin=1.7cm}
\setlist[enumerate]{leftmargin=1.7cm}

%--------------------------------------------------------------------------------
% compila o índice
%--------------------------------------------------------------------------------
\makeindex
% ---

%--------------------------------------------------------------------------------
% Comando para inserir imagens de forma simples
%--------------------------------------------------------------------------------
\newcommand{\imagem}[4]
{%			\imagem{x.x}{nomeimg}{titulo}{fonte}
	\begin{figure}[!htb]
		\caption{\label{img:#2}#3}
		\begin{center}
			\includegraphics[scale=#1]{img/#2}
		\end{center}
        \legend{\textbf{Fonte:} #4}
	\end{figure}
}%

%--------------------------------------------------------------------------------
% Creio que esses comandos sejam para desenhar algo, aguardando explicações de @leolleocomp
%--------------------------------------------------------------------------------
\newcommand{\xx} {$\bigotimes$}
\newcommand{\oo} {$\bigcirc$}

%--------------------------------------------------------------------------------
% Biblioteca para códigos-fonte
%--------------------------------------------------------------------------------
\usepackage[newfloat=true]{minted}

%--------------------------------------------------------------------------------
% Caixas batutas - by @leolleocomp
%--------------------------------------------------------------------------------
\usepackage[most]{tcolorbox}
\tcbuselibrary{breakable}

\tcbuselibrary{minted}
\tcbset{listing engine=minted}

\definecolor{bg}{rgb}{0.95,0.95,0.95}

\SetupFloatingEnvironment{listing}{name=Código, listname=Lista de códigos}

%--------------------------------------------------------------------------------
% configuração do contador dos códigos-fonte - by @leolleocomp
% assim como as figuras, começa em 1
\newcounter{sourcecode}
%--------------------------------------------------------------------------------
%--------------------------------------------------------------------------------
% @leolleocomp
% stackoverflow code
% peguei da resposta abaixo
% https://stackoverflow.com/questions/24086366/change-latex-minted-listings-numbering-to-include-current-section?answertab=votes#tab-top
%--------------------------------------------------------------------------------
\makeatletter
\renewcommand*{\thelisting}{\thesourcecode}
\makeatother

%--------------------------------------------------------------------------------
% Peçam explicações a @leolleo
% WHO DID THIS?
%--------------------------------------------------------------------------------
\newcommand{\Ididthis}{
%	\legend{\textbf{Fonte:} O autor (\the\year).}
\legend{\textbf{Fonte:} O autor}
}

\newcommand{\Otherguydidthis}[1]{
	\legend{\textbf{Fonte:} \citeonline{#1}.}
}

%--------------------------------------------------------------------------------
% Comando para inserir códigos - by @leolleocomp
%--------------------------------------------------------------------------------
\newcommand{\sourcecode}[4]{
\begin{listing}[H]
	\refstepcounter{sourcecode}
	\caption{#1}
	\label{cmd:#2}
	\inputminted[linenos, bgcolor=bg, tabsize=4,breaklines]{#3}{codes/#4}
	\Ididthis
\end{listing}
}


% -----------------------------------------------------------------------------
% Pacotes adicionados por @ruanmed
% ----------------------------------------------------------------------------- 
\usepackage[binary-units=true]{siunitx}     %   Pacote de unidades do SI com unidades binárias


%--------------------------------------------------------------------------------
% Comando para inserir códigos - by @ruanmed
%--------------------------------------------------------------------------------
% \usepackage{caption}

% \newenvironment{code}{\captionsetup{type=listing}}{}
% \SetupFloatingEnvironment{listing}{name=Código}

% \newcommand{\sourcecodenolist}[4]{
% 	\begin{code}
% 	    \refstepcounter{sourcecode}
%         \captionof{listing}{#1 }
%         \label{code:#2}
%         \inputminted[linenos, bgcolor=bg, tabsize=4,breaklines]{#3}{codes/#4}
%         \Wedidthis
%     \end{code}
% }

% \newcommand{\sourcecodenolist}[4]{
% 	\refstepcounter{sourcecode}
% 	\captionof{listing}{#1 \label{cmd:#2}}
% 	\inputminted[linenos, bgcolor=bg, tabsize=4,breaklines]{#3}{codes/#4}
% 	\Wedidthis
% }

% \newcommand{\sourcecodenolist}[4]{
% 	\refstepcounter{sourcecode}
% 	\captionof{listing}{#1 \label{cmd:#2}}
% 	\inputminted[linenos, bgcolor=bg, tabsize=4,breaklines]{#3}{codes/#4}
% 	\Wedidthis
% }


\newcommand{\sourcecodeinline}[2]{
	\mintinline[linenos, bgcolor=bg, tabsize=4,breaklines]{#1}{#2}
}

%--------------------------------------------------------------------------------
% Outros pacotes adicionais para circuitos - by @ruanmed
%--------------------------------------------------------------------------------
\RequirePackage{tikz}
\usepackage{tikz}
\usepackage[siunitx]{circuitikz}			% para habilitar o desenho de circuitos
\usetikzlibrary{babel}
% ---

%  https://tex.stackexchange.com/questions/324951/how-can-i-draw-a-vector-diagram-that-illustrates-polar-and-rectangular-coordinat
\usetikzlibrary{angles, arrows.meta, quotes}


% Pacote para usar modulo de vetor || v || com \norm - https://tex.stackexchange.com/questions/43008/absolute-value-symbols
\usepackage{commath}

%--------------------------------------------------------------------------------
% Outros pacotes adicionais para Siglas - by @ruanmed
%--------------------------------------------------------------------------------
\usepackage{nomencl}
	\makenomenclature

% Pacote para fazer listas enumeradas na mesma linha
% http://www.texnia.com/archive/enumitem.pdf
% % \usepackage{enumerate}
% \usepackage[inline, shortlabels]{enumitem}

\newlist{inlinelist}{enumerate*}{1}
\setlist*[inlinelist,1]{%
  label=(\alph*),
}

% Pacote para inserir divisória diagonal em tabela
\usepackage{diagbox}

% Pacote para auxiliar em tabelas criando novas células
\usepackage{makecell}

\usepackage{pbox,ragged2e}



% Gráficos
\usepackage{tikz}
\usepackage{pgfplots}
\pgfplotsset{compat=1.16}