\subsection{Ranqueamento} \label{subsec:Ranqueamento}
Devido à limitação dos métodos booleanos de somente retornar resultados conforme a presença ou não dos termos da consulta nos documentos \cite[p.~100]{Manning2008IIR}, foi proposto em 1957 por Luhn e 1959 por Maron \textit{et al.} uma abordagem de recuperação ranqueada \cite[p.~1446]{Sanderson2012THIRR} a qual, em contraste com recuperação booleana, baseada nos termos de consulta estabelecia uma pontuação para cada artigo de modo probabilístico e retornava os artigos de modo ordenado e demonstraram que essa técnica sobressaía a recuperação booleana.

% Explicar o funcionamento de uma recuperação ranqueada, como é feito esse ranqueamento?
O procedimento fundamental para ranqueamento dos documentos, conforme os termos de consulta, consiste na atribuição de pontuação aos documentos a partir da contabilização do número de aparições (chama de frequência) de cada um dos termos no documento.
Essa pontuação é calculada considerando que além da frequência do termo, denotada como $\text{tf}_{\text{\textit{t},\textit{d}}}$ que é o número de ocorrências do termo \textit{t} em um documento \textit{d}, existe também a sua relevância, que depende do número de aparições do termo na coleção de documentos inteira.
Quanto mais um termo aparece na coleção menos relevante ele é, e este valor de relevância é denotado por $\text{idf}_{\text{\textit{t}}}$ que é o inverso da frequência de um termo \textit{t} em uma coleção de documentos.
Segundo \citeonline[p.~108]{Manning2008IIR} este valor da relevância é calculado do seguinte modo:

\begin{equation}
    \text{idf}_{\text{\textit{t}}} = \log{\frac{N}{\text{df}_{\text{\textit{t}}}}}.
\end{equation}

O valor resultante da relação entre a frequência do termo e o inverso da frequência nos documentos é chamado de $\text{tf-idf}_{\text{\textit{t},\textit{d}}}$ (\textit{term frequency-inverse document frequency}), sendo este valor um dos pesos mais utilizados para ranqueamento \cite[p.~107--110]{Manning2008IIR}, e é calculado  como segue:
\begin{equation}
    \text{tf-idf}_{\text{\textit{t},\textit{d}}}  = \text{tf}_{\text{\textit{t},\textit{d}}} \times \text{idf}_{\text{\textit{t}}}.
\end{equation}

\begin{table}[h]
    \centering
    \begin{tabular}{|l|r|r|r|r|r|r|r|r|}\hline
         \multicolumn{3}{|c|}{\diagbox{Termo}{\raisebox{-1.87cm}{\rotatebox{90}{\parbox{1.6cm}{\centering Documento}}}}} & \multicolumn{2}{|c|}{Doc1} & \multicolumn{2}{|c|}{Doc2} & \multicolumn{2}{|c|}{Doc3} \\ \hline
                    & $\text{df}_{\text{\textit{t}}}$ & $\text{idf}_{\text{\textit{t}}}$ & $\text{tf}_{\text{\textit{t},\textit{d}}}$ & $\text{tf-idf}_{\text{\textit{t},\textit{d}}}$ & $\text{tf}_{\text{\textit{t},\textit{d}}}$ & $\text{tf-idf}_{\text{\textit{t},\textit{d}}}$ & $\text{tf}_{\text{\textit{t},\textit{d}}}$ & $\text{tf-idf}_{\text{\textit{t},\textit{d}}}$ \\ \hline
         car        & 18165 & 1,65 & 27 & 44,55 & 4 & 6,6 & 24 & 39,6 \\
         auto       & 6723 & 2,08 & 3 & 6,24 & 33 & 68,64 & 0 & 0 \\
         insurance  & 19241 & 1,62 & 0 & 0 & 33 & 54,46 & 29 & 46,98 \\
         best       & 25235 & 1,5 & 14 & 21 & 0 & 0 & 17 & 25,5
    \end{tabular}
    \caption{Exemplo de cálculo do valor de tf-idf baseado nas tabelas disponível em \citeonline[p.~109--110]{Manning2008IIR}.}
    \label{tab:exemplo-tf-idf}
\end{table}

Na Tabela \ref{tab:exemplo-tf-idf} temos um exemplo de cálculo dos valores de tf-idf para posterior cálculo da pontuação para ranqueamento, conforme alguma determinada consulta. 
A pontuação de um documento \textit{d} é a soma dos pesos de tf-idf de cada termo \textit{t} em \textit{d}, sendo os termos \textit{t} presentes na consulta realizada \cite[p.~109]{Manning2008IIR}, representamos esse cálculo do seguinte modo:

\begin{equation}
    \label{eq:pontuação-simples-tf-idf}
    \text{Pontuação(\textit{q},\textit{d})} = \sum_{\textit{t} \in \textit{q}}^{} \text{tf-idf}_{\text{\textit{t},\textit{d}}}.
\end{equation}

% Quando uma consulta é feita são utilizados os valores 
% (onde Pontuação({\textit{auto},\textit{car}},DocX))  

Utilizando a Equação \ref{eq:pontuação-simples-tf-idf} uma consulta com os termos \textit{auto car} retornaria no seu ranqueamento os documentos com a seguinte pontuação, calculamos Pontuação(\{\textit{auto},\textit{car}\},D\textsubscript{x}) para cada documento, por exemplo:
\begin{itemize}
    \setlength\itemsep{-0.2em}
    \item D\textsubscript{1}: 50,79
    \item D\textsubscript{2}: 75,24
    \item D\textsubscript{3}: 39,60
\end{itemize}

A ordenação dos documentos apresentados como resultado à consulta \textit{auto car} seria então a seguinte: 1\textordmasculine{} - D\textsubscript{2}; 2\textordmasculine{} - D\textsubscript{1}; e 3\textordmasculine{} - D\textsubscript{3}, que se observamos a Tabela \ref{tab:exemplo-tf-idf} é um bom resultado, já que o D\textsubscript{2} contém uma grande frequência do termo \textit{auto} e o D\textsubscript{3} não possui este termo.


Ao longo dos anos foi demonstrada a superioridade da recuperação ranqueada sobre a recuperação booleana \cite{Jones:1981:IRE:539571}, e são as técnicas de recuperação ranqueadas que trazem maior interesse para a área de Mineração de Textos, em específico estamos interessados nos modelos vetoriais e os modelos probabilísticos de RI que são evoluções da recuperação ranqueada.
% ESTOU PENSANDO EM JÁ CITAR O BM25 no parágrafo acima.
% -- Em algum ponto mencionar que Recuperação de Informação (Information Retrieval) não deve ser confundido com Procura de Informação (Information Search), pois a Procura de Informação é o campo que estuda a interação das pessoas com sistemas de recuperação de informação.

\subsubsection{Modelo de espaço vetorial} \label{subsubsec:Modelo-espaço-vetorial}
% O modelo de espaço vetorial.

% Modelo vetorial

%   	Tirado da cabeça
%     		Utiliza do vetor gerado pelo cálculo dos tf-idf calculados para cada palavra a partir da coleção inteira, e então define que o nível de similaridade entre uma consulta e um arquivo é dado pelo cosseno do ângulo entre os dois vetores.

%   	Do livro do Baeza
    A modelo vetorial surge a partir das limitações do modelo Booleano, que não considera frequência dos termos, e ele oferece a possibilidade de resultados parciais por meio da atribuição de pesos não binários para os termos da consulta e também para os termos presentes nos documentos, que são utilizados para determinar o grau de similaridade entre cada documento armazenado no sistema e uma determinada consulta.
    
    Neste modelo os resultados são apresentados em ordem decrescente de similaridade, considerando a correspondência parcial, e não obrigatoriamente total, com os termos da consulta, provendo assim uma resposta mais precisa para as necessidades de informação do usuário.
    
    Os pesos $w_{i,j}$ associados com um par termo-documento são não negativos e não binários. 
    Os termos de indexação são considerados mutualmente independentes e são representados como vetores unitários de um espaço t-dimensional, aonde t é número total de termos de indexação.
    A representação de um documento $d_j$ e uma consulta q são vetores t-dimensionais representandos como segue nas Equações abaixo:
    \begin{equation}
    			\vec{d_j} = (w_{1,j}, w_{2,j}, \cdots , w_{t,j})
    \end{equation}
    \begin{equation}
    			\vec{q} = (w_{1,q}, w_{2,q}, \cdots , w_{t,q})
    \end{equation}
    
    Aonde $w_{i,q}$ é o peso associado com o par termo-consulta $(k_i, q)$.
    
    Logo, tanto o documento $d_j$ quanto uma consulta $q$ feita pelo usuário são representados como vetores t-dimensionais como ilustrado na Figura \ref{fig:similaridade-cosseno}. 
    
    \begin{figure}[ht]
    \centering
    \caption{O cosseno de $\theta$ é adotado como $\text{Pontuação}(\vec{d_j},\vec{q})$. }
    \begin{center}
        \begin{tikzpicture}[scale=1.25]
            \coordinate[label=below  left:""] (A) at (-1.5cm,-1.cm);
            \draw[-Stealth]
                (A) to ["",sloped] ++ ({atan(2/3)}:{sqrt(4+9)})  coordinate[label=above right:$d_j$] (B);
            \draw[-Stealth] 
                (A) to ["" ']      ++ (0:3)                      coordinate[label=below right:q] (C);
            \pic ["$\theta$", draw, angle radius = 12mm, angle eccentricity=1.3]     
                {angle = C--A--B};
        \end{tikzpicture}
    \end{center}
    \vspace{-0.5cm}
    \legend{\ABNTEXfontereduzida \textbf{Fonte:} Baseado na figura disponível em \citeonline[p.~78]{Baeza-Yates2011}.}
    \label{fig:similaridade-cosseno}
\end{figure}

% \begin{figure}[ht]
%     \centering
%     \includegraphics[width=1.0\textwidth]{img/estrategico.png}
%     \caption{Planejamento estratégico da BSS Tecnologias.}
%     \label{fig:similaridade-cosseno}
% \end{figure}
    
    E então é feita a avaliação do grau de similaridade entre esses vetores, sendo esta correlação, ou a pontuação que o documento $d_j$ vai receber para a consulta $q$, quantificada pelo cosseno do ângulo entre esses dois vetores, conforme demonstra a Equação .
    \begin{equation}
    			\text{Pontuação}(\vec{d_j}, \vec{q}) = \frac{\vec{d_j} \bullet \vec{q} }{ \norm{\vec{d_j}} \times \norm{\vec{q}} }
    \end{equation}




\subsubsection{Modelo probabilístico}  \label{subsubsec:Modelo-probabilístico}
        
    
