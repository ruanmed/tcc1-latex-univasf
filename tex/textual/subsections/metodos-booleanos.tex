% -- Os primeiros sistemas de RI, booleanos 
\subsection{Métodos booleanos} \label{subsec:MétodosBooleanos}

    Uma consulta (chamada de \textit{query}) representa uma necessidade de informação a ser saciada por um sistema de RI, e essa consulta é composta de termos (um sinônimo para palavras) que nos primeiros desses sistemas era limitada a combinações lógicas e eram recuperados os documentos que tinham correspondência exata com ela \cite[p.~1446]{Sanderson2012THIRR}. 
    Esse método de recuperação de informação é conhecido como recuperação booleana, e para indexar os documentos é utilizada, geralmente, uma matriz binária de incidência de termo-documento. 
    Exemplificamos uma matriz dessas na Tabela \ref{tab:matriz-incidência-termo-documento}, que é o exemplo dado por \citeonline[p.~3--4]{Manning2008IIR} de uma matriz de incidência de termo-documento para o livro \textit{Shakespeare’s Collected Works}, que reúne as obras completas de Shakespeare.

    \begin{table}[H]
    \centering
    \caption{Matriz de incidência de termo-documento do livro \textit{Shakespeare’s Collected Works}. Cada elemento (i, j) da matriz é 1 se a peça de teatro na coluna j contém a palavra na linha i, caso contrário o elemento é 0.}
    \begin{adjustbox}{max width={\textwidth},keepaspectratio}%
    \begin{tabular}{|c|c|c|c|c|c|c|c}
        \hline
        \diagbox{Palavra}{
            \raisebox{-1.27cm}{
                \rotatebox{90}{
                    \parbox{1.6cm}{\centering Peça \\ de teatro}
                }
            }
        } 
        & \makecell{Antony \\ and \\ Cleopatra} 
        & \makecell{Julius \\ Caesar} 
        & The Tempest 
        & Hamlet 
        & Othello 
        & Macbeth 
        & ... 
        \\ \hline
        Antony     & 1 & 1 & 0 & 0 & 0 & 1 & \\
        Brutus     & 1 & 1 & 0 & 1 & 0 & 0 & \\
        Caeasar    & 1 & 1 & 0 & 1 & 1 & 1 & \\
        Calpurnia  & 0 & 1 & 0 & 0 & 0 & 0 & \\
        Cleopatra  & 1 & 0 & 0 & 0 & 0 & 0 & \\
        mercy      & 1 & 0 & 1 & 1 & 1 & 1 & \\
        worser     & 1 & 0 & 1 & 1 & 1 & 0 & \\
        ...        & & & & & & & 
    \end{tabular}
    \end{adjustbox}
    \legend{\ABNTEXfontereduzida \textbf{Fonte:} Tabela adaptada de \citeonline[p.~4]{Manning2008IIR}.}
    \label{tab:matriz-incidência-termo-documento}
\end{table}
% \clearpage