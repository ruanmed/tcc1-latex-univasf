%--------------------------------------------------------------------------------------
% Este arquivo contém a sua conclusão
%--------------------------------------------------------------------------------------
\chapter{Considerações Finais e Trabalhos Futuros} \label{ch:ConsideraçõesFinais}

% recapitular tudo, qual o propósito
% 1 parágrafo
Esse estudo investigativo buscou avaliar o desempenho de atributos oriundos da área de Recuperação de Informação em tarefas de Mineração de Texto, com enfoque em atributos gerados pela função BM25, por meio de uma metodologia própria.
A geração desses atributos foi subsidiada por ferramentas que implementam a função BM25 e o desempenho dos atributos de RI foi avaliado em dois corpus distintos.
O desempenho das ferramentas de armazenamento e indexação também foi avaliado.
% falar o que vi de fato nos resultados
% -- foi visto que para a base X, Y aconteceu provavelmente  devido a Z
 
Os resultados obtidos com a metodologia proposta mostram que o Zettair foi a ferramenta mais rápida na questão de indexação e também para consultas com textos de tamanho limitado.
O Elasticsearch também apresentou bom desempenho, não possuindo limitações de tamanho do texto nas consultas, e foi o melhor dentre os dois SGBDs testados.
Além disso, ele possui muito mais funcionalidades que o Zettair, que por sua vez não é um SGBD.

Ainda na avaliação de desempenho das ferramentas, percebe-se que o ArangoDB obteve um resultado muito aquém dos demais na medida TIME\_QUERY para o corpus de 300 mil registros, indicando que as consultas utilizando a função BM25 do ArangoDB são lentas para bancos de dados com grande número de documentos.
% reforçar as melhores ferramentas, porque são resultados de desempenho mais claros

Dentre as soluções selecionadas para reprodução, só foi possível comprovar as medidas do ranking final das competições para a solução 2\_daneshvar18 do corpus DB\_AUTHORPROF, para as duas soluções do corpus DB\_HYPERPARTISAN isso não foi possível pois o conjunto de validação final da competição não foi publicado.
Contudo, as soluções do DB\_HYPERPARTISAN selecionadas mantiveram a relação de posição quando reproduzidas, com a solução 1\_bertha obtendo melhor acurácia e melhor F1-Score do que a solução 4\_tom.

Na avaliação de desempenho dos classificadores adicionando os atributos de RI, a Rede Neural Convolucional (1\_bertha) apresentou o maior ganho de desempenho, saltando de uma acurácia de $81,40\%$ para $86,51\%$ e $F_1$-Score de $76,19\%$ para $81,05\%$ ao utilizar o ArangoDB para gerar os atributos de RI.
A adição dos atributos de RI gerados por quaisquer ferramentas proporcionou ganho de desempenho à CNN da solução 1\_bertha para o corpus DB\_HYPERPARTISAN, porém, não houve qualquer ganho de desempenho ao adicionar os atributos de RI ao classificador SVC da solução 4\_tom.

Uma análise e seleção dos parâmetros do classificador SVC mostrou que adicionar os atributos de RI gerados pelo Zettair produz um ganho de desempenho de acurácia de $82,33\%$ para $84,19\%$ e $F_1$-score de $73,61\%$ para $76,06\%$ quando o melhor valor do parâmetro C é selecionado.

Quanto ao corpus DB\_AUTHORPROF, o classificador LinearSVC da solução 2\_daneshvar18 não obteve nenhum ganho de desempenho ao serem adicionados os atributos de RI, nem mesmo ao ser feita a análise e seleção do parâmetro C do classificador.
Uma investigação posterior revelou que na verdade a magnitude dos atributos de RI era bem maior que a dos atributos originais, prejudicando o classificador LinearSVC.
No entanto, mesmo com o reparo de magnitude por meio do escalonador MinMaxScaler, e nova análise e seleção do parâmetro C do classificador, não foi obtido nenhum ganho de desempenho ao adicionar os atributos de RI à solução 2\_daneshvar18.

Então, conclui-se que os 6 atributos de RI sugeridos neste estudo proporcionaram ganho de desempenho somente para classificações do corpus DB\_HYPERPARTISAN, um corpus pequeno com documentos extensos.
Os atributos de RI funcionaram melhor com o classificador CNN, e para o classificador SVC somente o Zettair proporcionou ganho de desempenho.
No corpus DB\_AUTHORPROF, um corpus grande com documentos curtos, com o classificador LinearSVC, os atributos de RI causaram perda de desempenho.
Para corroborar o ganho de desempenho, ou não, dos atributos de RI sugeridos são necessários mais testes pois os resultados aqui obtidos não são unânimes.
São necessários testes com outros classificadores nesses mesmos corpus e também em diferentes corpus, corpus grandes com documentos extensos e corpus pequenos com documentos pequenos. 
Podem ser pensados também testes com diferentes valores de top-$k$ e parâmetros $k_1$, $k_3$ e $b$ para cálculo dos atributos de RI.

% limitações do que foi feito

% 
\section{Trabalhos futuros}
    Para trabalhos futuros está planejada uma maior análise exploratória dos parâmetros da solução 2\_daneshvar18, reproduzindo a solução com maiores variações nos parâmetros do classificador LinearSVC, inclusive com alteração do núcleo do mesmo, para verificar se alguma configuração dele possui desempenho superior constante ao serem adicionados os atributos de RI. 
    Isso não foi feito neste estudo devido à limitação temporal para executar todas as variações necessárias para chegar a mais conclusões, mas alguns resultados já estão guardados.
    % mas algumas das diferentes configurações já foram executadas ou reproduzidas

    Quanto maior o número de soluções reproduzidas para cada corpus melhor, mas trabalhar com um mínimo talvez seja o ideal, então para todas as reproduções futuras é bom tomar como base no mínimo duas soluções, ou classificadores distintos, por corpus, para efetuar as comparações.
    Então, em um trabalho futuro imediato planeja-se reproduzir mais uma das soluções do corpus DB\_AUTHORPROF disponíveis online.

    Como mencionado antes, trabalhos futuros podem também trabalhar somente analisando diferentes valores que influenciem somente nos atributos de RI gerados, o top-$k$ e os parâmetros $k_1$, $k_3$ e $b$ reproduzindo-os para os mesmos corpus.

    Além disso, uma outra possibilidade de estudo, é o maior isolamento do índice de geração dos atributos de RI do conjunto de treinamento por meio da separação dos corpus em 3 pedaços.
    Nessa separação o primeiro pedaço seria utilizado para para treinamento do classificador, o segundo pedaço seria utilizado como o índice para geração dos atributos derivados da função BM25 para o primeiro pedaço, e o terceiro pedaço ficaria para teste/validação do classificador.

% \lipsum[55]