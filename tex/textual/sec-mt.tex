\section{Mineração de Texto} \label{sec:MineraçãoTexto}
% Introdução a Mineração de Texto
% - Falar de IA?
% -- Surgimento do processo de KDD
% -- É um subramo do KDD, KDT
A Mineração de Textos (MT) é às vezes tradada como \textit{knowledge discovery in text} (livremente traduzido para descoberta de conhecimento em texto) (Kodratoff1999KDTDA e Feldman1995KDT), sendo análogo ao termo \textit{knowledge discovery in data} (KDD) que se refere à Mineração de Dados, ramo da Inteligência Artificial que dá suporte a MT. 
Apesar de haver um uso sinônimo entre Mineração de Dados e KDD, alguns autores tratam a Mineração de Dados como somente uma parte do processo desse processo de descoberta de conhecimento (Han2011, p.6), sendo este um processo iterativo composto pelas seguintes fases segundo Han2011, p.6 e 7:
% - Passos da MT
% 1. Data cleaning (to remove noise and inconsistent data)
% 2. Data integration (where multiple data sources may be combined)3
% 3. Data selection (where data relevant to the analysis task are retrieved from the  database)
% 4. Data transformation (where data are transformed and consolidated into forms appropriate for mining by performing summary or aggregation operations)4
% 5. Data mining (an essential process where intelligent methods are applied to extract data patterns)
% 6. Pattern evaluation (to identify the truly interesting patterns representing knowledge  based on interestingness measures—see Section 1.4.6)
% 7. Knowledge presentation (where visualization and knowledge representation techniques are used to present mined knowledge to users)
\begin{enumerate}
    \item \textbf{Limpeza dos dados}: remoção de ruído e dados inconsistentes;
    \item \textbf{Integração dos dados}: combinação de múltiplas fontes de dados;
    \item \textbf{Seleção dos dados}: dados relevantes para a tarefa de análise são recuperados do banco de dados;
    \item \textbf{Transformação dos dados}: dados são transformados e consolidados em formas apropriadas para mineração sendo realizadas, por exemplo, ações de agregação ou resumo;
    \item \textbf{Mineração dos dados}: métodos inteligentes são aplicados para extrair padrões de dados;
    \item \textbf{Avaliação de padrões}: são identificados os padrões que realmente tão interessantes para representar o conhecimento baseado em medidas de nível de interesse;
    \item \textbf{Apresentação do conhecimento}: o conhecimento minerado é apresando aos usuários por meio de técnicas de visualização e representação de conhecimento.
\end{enumerate}

% Na década de 70 houve o surgimento de diversas técnicas de gerenciamento de banco de dados, como por exemplo a 

% -- Utiliza de várias áreas, falar delas
% -- Utiliza das técnicas de RI para indexar os textos em algumas de suas aplicações
% -- Já devo diferenciar aqui