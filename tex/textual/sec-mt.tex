\section{Mineração de Texto} \label{sec:MineraçãoTexto}
% Introdução a Mineração de Texto
% - Falar de IA?
% -- Surgimento do processo de KDD
% -- É um subramo do KDD, KDT
A Mineração de Textos (MT) é definida como o processo de extrair conhecimento implícito de dados textuais (Jo2018 e Feldman2007) e por isso é às vezes tradada como \textit{knowledge discovery in text} (livremente traduzido para descoberta de conhecimento em texto) (Kodratoff1999KDTDA e Feldman1995KDT), sendo análogo ao termo \textit{knowledge discovery in data} (KDD) que se refere à Mineração de Dados, ramo da Inteligência Artificial que dá suporte a MT. 
Apesar de haver um uso sinônimo entre Mineração de Dados e KDD, alguns autores tratam a Mineração de Dados como somente uma parte desse processo de descoberta de conhecimento (Han2011, p.6), sendo este um processo iterativo composto pelas seguintes fases segundo Han2011, p.6 e 7:
% - Passos da MT
% 1. Data cleaning (to remove noise and inconsistent data)
% 2. Data integration (where multiple data sources may be combined)3
% 3. Data selection (where data relevant to the analysis task are retrieved from the  database)
% 4. Data transformation (where data are transformed and consolidated into forms appropriate for mining by performing summary or aggregation operations)4
% 5. Data mining (an essential process where intelligent methods are applied to extract data patterns)
% 6. Pattern evaluation (to identify the truly interesting patterns representing knowledge  based on interestingness measures—see Section 1.4.6)
% 7. Knowledge presentation (where visualization and knowledge representation techniques are used to present mined knowledge to users)
\begin{enumerate}
    \item \textbf{Limpeza dos dados}: remoção de ruído e dados inconsistentes;
    \item \textbf{Integração dos dados}: combinação de múltiplas fontes de dados;
    \item \textbf{Seleção dos dados}: dados relevantes para a tarefa de análise são recuperados do banco de dados;
    \item \textbf{Transformação dos dados}: dados são transformados e consolidados em formas apropriadas para mineração sendo realizadas, por exemplo, ações de agregação ou resumo;
    \item \textbf{Mineração dos dados}: métodos inteligentes são aplicados para extrair padrões de dados;
    \item \textbf{Avaliação de padrões}: são identificados os padrões que realmente tão interessantes para representar o conhecimento baseado em medidas de nível de interesse;
    \item \textbf{Apresentação do conhecimento}: o conhecimento minerado é apresando aos usuários por meio de técnicas de visualização e representação de conhecimento.
\end{enumerate}
% Certainly, text mining derives much of its inspiration and direction from seminal research on data mining. Therefore, it is not surprising to find that text mining and data mining systems evince many high-level architectural similarities. 

É importante notar as etapas de desenvolvimento de Mineração de Dados para abordamos a definição de MT pois esta deriva muitas técnicas desenvolvidas na pesquisa do campo de Mineração de Dados para seu campo de aplicação, logo sistemas baseados em ambas áreas vão apresentar similaridades arquiteturais (Feldman2007). 

% Text mining can be broadly defined as a knowledge-intensive process in which a user interacts with a document collection over time by using a suite of analysis tools. In a manner analogous to data mining, text mining seeks to extract useful information from data sources through the identification and exploration of interesting patterns.

% Because data mining assumes that data have already been stored in a structured format, much of its preprocessing focus falls on two critical tasks: Scrubbing and normalizing data and creating extensive numbers of table joins. In contrast, for text mining systems, preprocessing operations center on the identification and extraction of representative features for natural language documents. These preprocessing operations are responsible for transforming unstructured data stored in document collections into a more explicitly structured intermediate format, which is a concern that is not relevant for most data mining systems.

A Mineração de Dados assume que os dados que vão ser tratados pelo seu processo já foram armazenados em um formato estruturado, logo a maior parte de seu pré-processamento vai estar ligado às etapas 1 e 2 do processo de KDD citado, as de limpeza e integração dos dados (Feldman2007). 
Já na MT, como os dados de trabalho são textos, sendo texto configurado como dados desestruturados que consistem de \textit{strings}, chamadas de palavras, organizados de forma coerente e sendo pertencentes a uma linguagem natural (Jo2018), temos que as operações de pré-processamento vão estar mais focadas em etapas adicionais, anteriores às citadas para o processo de KDD, que seriam focadas na identificação e extração de \textit{features} (variáveis) representativas para documentos escritos em linguagem natural, transformando os dados não estruturados, que estão armazenados em coleções de documentos, em um formato mais explicitamente estruturado (Feldman2007).

% Text is defined as the unstructured data which consists of strings which are called words [82]

% Na década de 70 houve o surgimento de diversas técnicas de gerenciamento de banco de dados, como por exemplo a 

% -- Utiliza de várias áreas, falar delas
% -- Utiliza das técnicas de RI para indexar os textos em algumas de suas aplicações
% -- Já devo diferenciar aqui