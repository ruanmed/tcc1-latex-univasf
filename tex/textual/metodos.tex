% ísticos que baseiam-se na %--------------------------------------------------------------------------------------
% Este arquivo contém a sua metodologia
%--------------------------------------------------------------------------------------
\chapter{Materiais e Métodos} \label{ch:MateriaisMétodos} %Uma label é como você referencia uma seção no texto com a tag \ref{}
Neste capítulo será apresentado como será feita a avaliação do desempenho das funções de ranqueamento na mineração de texto.

Para utilização da função de ranqueamento BM25

\section{Seção de exemplo 1} 
% Metologia:

% Vão ser elencadas as tecnologias para armazenamento e indexação dos dados por meio do BM25:
% * Elasticsearch
% * APache Spark
% * ArangoDB
% * Zettair

% Os bancos de teste serão:
% * PanCLEF Hyperpartisan 2019
% https://pan.webis.de/semeval19/semeval19-web/leaderboard.html
% * A definir
% * Bots and Gender Profilin % PAN @ CLEF 2019 


% O teste de desempenho será feito comparando as ferramentas de indexação, tempo para indexar o treino de cada banco (TIME_TRN) em cada ferramenta.

% O tempo para consultar para cada linha do teste as variáveis a serem criadas (no caso de classificação binária iremos definir agregações do tipo count, sum e avg). Por exemplo, para a classificação binária do PANCLEF 2019 Hyperpartisan podem ser criadas as variáveis a seguir:
% * avg_0 
% * count_0
% * sum_0
% * avg_1
% * count_1
% * sum_1

% Ao final será comparado o ganho de desempenho (acurácia, precisão, recall e F1-score) nas melhores soluções dos bancos definidos a partir da adição das variáveis criadas.
% Por exemplo para a PANCLEF 2019 Hyperpartisan será utilizada a solução que está disponível no github com melhor score e ela será executada novamente para conferir o resultado obtido de acurária que eles dizem e então será rodado novamente com a adição das variáveis de RI (BM25).


\subsection{Subseção de exemplo 1 - Referenciando seções} \label{subsec:subsec1}






%--------------------------------------------------------------------------------------
% Insere a seção de cronograma
% Está comentada porque só é necessária no TCC I
%--------------------------------------------------------------------------------------
% %--------------------------------------------------------------------------------------
% Insere a seção de cronograma
%--------------------------------------------------------------------------------------

\section{Cronograma} \label{sec:Cronograma}

A Tabela \ref{tab:cronograma} mostra o cronograma de atividades a serem executadas para o Trabalho de Conclusão II (TCC II), com base no calendário do período 2019.2 da UNIVASF, definido pelo Calendário Acadêmico 2019 da instituição.

\begin{table}[!thb]
	%\huge
    \centering
    \caption{Cronograma das atividades previstas para o TCC II.}
    \begin{adjustbox}{max width=\textwidth}
    \begin{tabular}{p{6.5cm}|c|c|c|c|c|c}
        \toprule
        \textbf{Atividade}
        & Set & Out & Nov & Dez & Jan & Fev
        \\ \hline
        Definição e obtenção dos corpus para avaliação 
        & X   &     &     &     &     &          
        \\ \hline
        Inspeção e seleção das soluções com código fonte disponível
        & X   &     &     &     &     &          
        \\ \hline
        Instalação e familiarização com as ferramentas de arquivamento e indexação
        & X   & X   &     &     &     &          
        \\ \hline
        Indexação do conjunto de treinamento dos corpus
        &     & X   & X   &     &     &          
        \\ \hline
        Adição dos atributos de RI às soluções selecionadas
        &     & X   & X   &     &     &    
        \\ \hline
        Mineração dos dados por meio da reprodução das soluções selecionadas com/sem adição dos atributos de RI
        &     &     & X   & X   & X   &  
        \\ \hline
        Escrita do TCC II                       
        & X   & X   & X   & X   & X   &         
        \\ \hline
        Defesa do TCC II                        
        &     &     &     &     & X   &        
        \\
        \bottomrule
    \end{tabular}
    \end{adjustbox}
    
    \label{tab:cronograma} 
    % \legend{\textbf{Fonte:} O autor.}
\end{table}
