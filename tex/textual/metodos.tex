% ísticos que baseiam-se na %--------------------------------------------------------------------------------------
% Este arquivo contém a sua metodologia
%--------------------------------------------------------------------------------------
\chapter{Materiais e Métodos} \label{ch:MateriaisMétodos} %Uma label é como você referencia uma seção no texto com a tag \ref{}
    Neste capítulo será apresentado como será feita a avaliação do desempenho das funções de ranqueamento na Mineração de Texto.
    
    Para utilização da função de ranqueamento BM25 como variável em tarefas de mineração de texto é necessário efetuar o armazenamento e a indexação dos textos do conjunto de treinamento, e para esta tarefa serão utilizadas ferramentas de armazenamento em banco de dados (não relacionais?) que fornecem implementações do BM25 para consulta, estas serão apresentadas na Seção \ref{sec:Armazenamento-e-indexação}.
    
    Se faz necessário também definir métricas para avaliação do ganho de desempenho com as variáveis de RI, assim como elencar as bases de dados que vão servir para efetuar essa avaliação, disposto logo mais nas Seções \ref{sec:Métricas-de-desempenho-para-avaliação} e \ref{sec:Bancos-de-dados-para-teste}.

\section{Armazenamento e indexação} \label{sec:Armazenamento-e-indexação}

    Para armazenar e indexar os bancos de dados das tarefas de Mineração de Textos, e possibilitar assim o cálculo da função BM25 para cada exemplo, serão utilizadas as seguintes tecnologias que fazem implementações do BM25:
    \begin{itemize}
        \item Elasticsearch % https://www.elastic.co/guide/en/elasticsearch/reference/current/index-modules-similarity.html
        % https://www.elastic.co/pt/blog/practical-bm25-part-2-the-bm25-algorithm-and-its-variables
        \item ArangoDB
        % https://www.arangodb.com/docs/stable/aql/views-arango-search.html#bm25
        % https://www.arangodb.com/news/introducing-arangodb-3-4/
        \item Zettair
        % http://www.seg.rmit.edu.au/zettair/doc/Readme.html
        \item Apache Spark
    \end{itemize}
    % Links dos sites das ferramentas como notas de rodapé.
    % Citar a versão de cada ferramenta que vai ser utilizada.
    % 
    
\section{Bancos de dados para teste}  \label{sec:Bancos-de-dados-para-teste}
    Os banco de dados de teste selecionados para avaliação são os utilizados na PanCLEF 2019 % necessário citar a PanCLEF anteriormente

    \begin{itemize}
        \item Bots and Gender Profiling PAN @ CLEF 2019
        % https://pan.webis.de/clef19/pan19-web/author-profiling.html
        \item Hyperpartisan News Detection - PAN @ SemEval 2019
        %  https://pan.webis.de/semeval19/semeval19-web/index.html
    \end{itemize}
    % Descreuioes a partir da tarefa no site
    % Definir as soluçcoes disponiveis que vou utilizar (Nao pode ter RI)
 
% Metodologia:

% Vão ser elencadas as tecnologias para armazenamento e indexação dos dados por meio do BM25:
% * Elasticsearch
% * APache Spark
% * ArangoDB
% * Zettair

% Os bancos de teste serão:
% * PanCLEF Hyperpartisan 2019
% https://pan.webis.de/semeval19/semeval19-web/leaderboard.html
% * A definir
% * Bots and Gender Profilin % PAN @ CLEF 2019 

\section{Variáveis de RI sugeridas}  \label{sec:Variáveis-de-RI-sugeridas}
    
%  Por exemplo, para a classificação binária do PANCLEF 2019 Hyperpartisan podem ser criadas as variáveis a seguir:
% * avg_0 
% * count_0
% * sum_0
% * avg_1
% * count_1
% * sum_1

% baseadas no weren (2014)


\section{Métricas para avaliação de desempenho}  \label{sec:Métricas-para-avaliação-de-desempenho}
    Será avaliado o desempenho das ferramentas utilizadas para indexação e consulta, por meio da avaliação temporal do desempenho computacional, e também o ganho de desempenho propiciado pelo uso das variáveis de RI em termos das métricas de classificadores de Mineração de Texto. % Essas métricas tem que ser citadas no referencial teórico.
    
    \subsection{Desempenho computacional das ferramentas}  \label{sec:Desempenho-computacional}
        Para avaliar o desempenho das ferramentas de armazenamento e indexação utilizaremos duas métricas temporais:
        \begin{itemize}
            \item \textbf{TIME\_INDEX}: Tempo de execução para indexar o conjunto de treinamento de cada um dos banco de dados de teste elencados na Seção \ref{sec:Bancos-de-dados-para-teste}. Dadas as quatro diferentes ferramentas de indexação e os dois banco de dados selecionados, serão computadas 8 TIME\_INDEX para comparação;
            
            \item \textbf{TIME\_QUERY}: Tempo para consulta de cada exemplo do conjunto de teste e geração das variáveis sugeridas na Seção \ref{sec:Variáveis-de-RI-sugeridas} para o item específico. Dadas as 12 variáveis sugeridas para criação distribuídas nos dois banco de dados selecionados, e dadas as 4 ferramentas de indexação, 48 TIME\_QUERY serão computadas.  
        \end{itemize}
        
        Essas variáveis serão computadas para cada uma das 4 ferramentas selecionadas sendo executadas no mesmo sistema computacional a fim de oferecer maior confiança aos números obtidos. 
        O sistema computacional a ser utilizado para efetuar o experimento ainda não está definido, mas será especificado nos resultados.
    
% O teste de desempenho será feito comparando as ferramentas de indexação, tempo para indexar o treino de cada banco (TIME_TRN) em cada ferramenta.

% O tempo para consultar para cada linha do teste as variáveis a serem criadas (no caso de classificação binária iremos definir agregações do tipo count, sum e avg). Por exemplo, para a classificação binária do PANCLEF 2019 Hyperpartisan podem ser criadas as variáveis a seguir:
% * avg_0 
% * count_0
% * sum_0
% * avg_1
% * count_1
% * sum_1
    \subsection{Desempenho de classificador}  \label{sec:Desempenho-de-classificador}
    O ganho de desempenho propiciado pelas variáveis de RI criadas será mensurado por métricas de desempenho de classificadores da Mineração de Dados, as mesmas da Mineração de Texto.
    Nesse estudo serão computadas as seguintes métricas:
    \begin{itemize}
        \item \textbf{CLF\_ACC}: Acurácia do classificador.
        \item \textbf{CLF\_AUC}: Área sobre a curva ROC.
        \item \textbf{CLF\_F1}: F1-Score
        \item \textbf{CLF\_PRE}: Precisão do classificador.
        \item \textbf{CLF\_REC}: Revocação do classificador.
    \end{itemize}
% Ao final será comparado o ganho de desempenho (acurácia, precisão, recall e F1-score) nas melhores soluções dos bancos definidos a partir da adição das variáveis criadas.
% Por exemplo para a PANCLEF 2019 Hyperpartisan será utilizada a solução que está disponível no github com melhor score e ela será executada novamente para conferir o resultado obtido de acurária que eles dizem e então será rodado novamente com a adição das variáveis de RI (BM25).


ADICIONAR IMAGEM DE VISÃO GERAL DA METODOLOGIA


% \subsection{Subseção de exemplo 1 - Referenciando seções} \label{subsec:subsec1}






%--------------------------------------------------------------------------------------
% Insere a seção de cronograma
% Está comentada porque só é necessária no TCC I
%--------------------------------------------------------------------------------------
% %--------------------------------------------------------------------------------------
% Insere a seção de cronograma
%--------------------------------------------------------------------------------------

\section{Cronograma} \label{sec:Cronograma}

A Tabela \ref{tab:cronograma} mostra o cronograma de atividades a serem executadas para o Trabalho de Conclusão II (TCC II), com base no calendário do período 2019.2 da UNIVASF, definido pelo Calendário Acadêmico 2019 da instituição.

\begin{table}[!thb]
	%\huge
    \centering
    \caption{Cronograma das atividades previstas para o TCC II.}
    \begin{adjustbox}{max width=\textwidth}
    \begin{tabular}{p{6.5cm}|c|c|c|c|c|c}
        \toprule
        \textbf{Atividade}
        & Set & Out & Nov & Dez & Jan & Fev
        \\ \hline
        Definição e obtenção dos corpus para avaliação 
        & X   &     &     &     &     &          
        \\ \hline
        Inspeção e seleção das soluções com código fonte disponível
        & X   &     &     &     &     &          
        \\ \hline
        Instalação e familiarização com as ferramentas de arquivamento e indexação
        & X   & X   &     &     &     &          
        \\ \hline
        Indexação do conjunto de treinamento dos corpus
        &     & X   & X   &     &     &          
        \\ \hline
        Adição dos atributos de RI às soluções selecionadas
        &     & X   & X   &     &     &    
        \\ \hline
        Mineração dos dados por meio da reprodução das soluções selecionadas com/sem adição dos atributos de RI
        &     &     & X   & X   & X   &  
        \\ \hline
        Escrita do TCC II                       
        & X   & X   & X   & X   & X   &         
        \\ \hline
        Defesa do TCC II                        
        &     &     &     &     & X   &        
        \\
        \bottomrule
    \end{tabular}
    \end{adjustbox}
    
    \label{tab:cronograma} 
    % \legend{\textbf{Fonte:} O autor.}
\end{table}
