\section{Recuperação de Informação} \label{sec:RecuperaçãoInformação}
% Falar da recuperação de informação com conceito humano

% Estrutura

% - Histórico de RI, explicando o surgimento de métodos de RI com as bibliotecas


A procura por informação é uma necessidade das pessoas, e um dos modos bem eficientes de assim fazer é consultar outras pessoas por informação, no entanto devido ao grande acumulo de informação das sociedades, uma só pessoa não pode carregar consigo todo o conhecimento da humanidade, para tanto um modo eficiente de transferir esse conhecimento, que tratamos aqui como informação, é por meio de registros físicos em papel, livros e similares.

No intuito de organizar esses diversos livros, artigos e afins, é notária a função dos bibliotecários em separar os diversos tipos de conhecimento que os mais diversos livros podem abrir, e portanto assim os sistemas de classificação de áreas, e subáreas, do conhecimento são um auxílio para as necessidades de busca por informação que uma pessoa pode ter. Esses sistemas de classificação facilitam a localização de informações específicas a partir de uma pergunta que uma pessoa pode fazer, no entanto é necessário o conhecimento de como o sistema de classificação funciona para que um usuário possa tentar saciar sua necessidade, ou pelo menos de um especialista no sistema (o bibliotecário) para lhe guiar, e mesmo assim depois de adquirir os diversos materiais que podem responder à sua pergunta, o mesmo ainda terá que conferir nos textos se ele satisfaz.

% - RI em equipamentos eletromecânicos no início do século 20
% -- O que fez surgir esses sistemas, a necessidade

Esses sistemas de classificação manual se mostram bem ineficazes devido ao surgimento crescente constante de novas informações desde o início do século 20, temos assim uma imensidão de novas pesquisas científicas sendo publicadas, livros surgindo, entre outros registro históricos, que precisam ser classificados, mas existe também o problema da classificação não poder abranger todo tipo de necessidade que tal material pode satisfazer. 
A preocupação com sistemas que possam indexar todo esse material e fornecer um acesso rápido às necessidades de informação de uma pessoa surgiu também no início do século 20 (CITAR ARTIGO DO MEMEX AQUI), onde sistemas mecânicos de recuperação de informação foram visionados.


% - A área de pesquisa de RI que surgiu com a utilização de computadores para indexar informação
% -- A utilização de computadores para fazer o serviço de RI

A partir da criação de sistemas computacionais na década de 1940, foi vista a possibilidade de criação de sistemas que armazenassem informações e possibilitassem essa consulta rápida sobre as informações armazenadas, sendo necessários estabelecer algoritmos que retornassem informação relevante ao que usuário do sistema procura. 
Tendo então aqui o início do campo científico de Recuperação de Informação (\textit{Information retrieval}, IR) que é encontrar material (geralmente documentos) de natureza desestruturada (geralmente texto) que satisfaça uma necessidade de informação dentro de grandes acervos (geralmente armazenados em computadores). (CITAÇAO DO LIVRO C D MANNING).

A necessidade de Sistemas de Recuperação de Informação (SRI) surge do crescimento exponencial das coleções de informação, e consequente inabilidade das técnicas tradicionais de catalogação de lidar com isso (SANDERSON). 
A lei de Moore diz que o crescimento da velocidade de processamento é contínua, e similarmente existe uma duplicação constante da capacidade de armazenamento digital a cada dois anos. 
Ter um amontoado de conhecimento, informação, e não poder acessar o que é relevante de modo rápido não é interessante pois assim o desenvolvimento de pesquisas, por exemplo, fica comprometido e pode perder relevância (CITAR ARTIGO DE 1945).

% -- Conceitos fundamentais de RI
A RI como uma disciplina de pesquisa se iniciou no final da década de 50 com o início do uso de computadores para procurar referências de texto associadas com um assunto (SANDERSON, p.3), as preocupações iniciais dessa área são \textit{como indexar documentos} e \textit{como recuperá-los}, sendo a busca como melhor fazer essas tarefas o objetivo da RI.

Logo no início do seu desenvolvimento as técnicas de RI buscaram se basear em sistemas existentes já consolidados no campo bibliotecário para indexar coleções de itens, tendo como uma técnica de abordagem clássica atribuir códigos numéricos a essas coleções, como por exemplo o feito pelo sistema de Classificação Decimal de Dewey (CITA SANDERSON, p. 3), no entanto foi demonstrado por Cleverdon que um sistema baseado em palavras, como o sistema Uniterm proposto por Taube et al (SANDERSON, p.3), era tão bom e até melhor que outras abordagens clássicas, sendo posteriormente adotado pelos SRI a indexação por palavras (SANDERSON, p.3).

% -- Os primeiros sistemas de RI, booleanos 
Uma \textit{query} (ou pesquisa) representam uma necessidade de informação a ser saciada por um SRI, e essa \textit{query} é composta de termos (um sinônimo para palavras) que nos primeiros desses sistemas era limitada a combinações lógicas e eram recuperados os documentos que tinham correspondência exata com ela. 
No entanto, foi proposto em XXXX por Luhn e Maron (CITAR tá no SANDERSON) uma abordagem de recuperação ranqueada, em contraste com recuperação booleana, a qual, com base nos termos de \textit{query}, estabelecia um pontuação para cada artigo de modo probabilistico e retornava os artigos de modo ordenado e demonstraram que essa técnica sobressaía a recuperação booleana. 
Ao longo dos anos foi demonstrada a superioridade da recuperação ranqueada sobre a recuperação booleana (CITAR ARTIGO 22 do SANDERSON), são essas técnicas de recuperação ranqueadas que trazem maior interesse para a área de Mineração de Textos.

% -- A pesquisa de RI sendo desenvolvida, citar sistemas rankeados
% -- Em algum ponto mencionar que Recuperação de Informação (Information Retrieval) não deve ser confundido com Procura de Informação (Information Search), pois a Procura de Informação é o campo que estuda a interação das pessoas com sistemas de recuperação de informação.


% TESTE DE SIGLAS
\nomenclature{Fig.}{Figura TTTTT}
\nomenclature{$A_i$}{Area of the $i^{th}$ component}
\nomenclature{456}{Isto éum número}
\nomenclature{123}{Isto éoutro número}
\nomenclature{a}{primeira letra do alfabeto}
\nomenclature{lauro}{este émeu nome}