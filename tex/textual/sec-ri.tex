\section{Recuperação de Informação} \label{sec:RecuperaçãoInformação}
% Falar da recuperação de informação com conceito humano

A procura por informação é uma necessidade das pessoas, e um dos modos bem eficientes de assim fazer é consultar outras pessoas por informação, no entanto devido ao grande acumulo de informação das sociedades, uma só pessoa não pode carregar consigo todo o conhecimento da humanidade, para tanto um modo eficiente de transferir esse conhecimento, que tratamos aqui como informação, é por meio de registros físicos em papel, livros e similares.

No intuito de organizar esses diversos livros, artigos e afins, é notária a função dos bibliotecários em separar os diversos tipos de conhecimento que os mais diversos livros podem abrir, e portanto assim os sistemas de classificação de áreas, e subáreas, do conhecimento são um auxílio para as necessidades de busca por informação que uma pessoa pode ter. Esses sistemas de classificação facilitam a localização de informações específicas a partir de uma pergunta que uma pessoa pode fazer, no entanto é necessário o conhecimento de como o sistema de classificação funciona para que um usuário possa tentar saciar sua necessidade, ou pelo menos de um especialista no sistema (o bibliotecário) para lhe guiar, e mesmo assim depois de adquirir os diversos materiais que podem responder à sua pergunta, o mesmo ainda terá que conferir no texto se ele satisfaz.

Esses sistemas de classificação manual se mostram bem ineficazes devido ao surgimento crescente constante de novas informações desde o início do século 20, temos assim uma imensidão de novas pesquisas científicas sendo publicadas, livros surgindo, entre outros registro históricos, que precisam ser classificados, mas existe também o problema da classificação não poder abranger todo tipo de necessidade que tal material pode satisfazer. A preocupação com sistemas que possam indexar todo esse material e fornecer um acesso rápido às necessidades de informação de uma pessoa surgiu também no início do século 20 (CITAR ARTIGO DO MEMEX AQUI), onde sistemas mecânicos de recuperação de informação foram visionados.

A partir da criação de sistemas computacionais na década de 1940, foi vista a possibilidade de criação de sistemas armazenassem informações e possibilitassem essa consulta rápida sobre as informações armazenadas, sendo necessários estabelecer algoritmos que retornassem informação relevante ao que usuário do sistema procura. Tendo então aqui o início do campo científico de Recuperação de Informação (Information retrieval, IR) que é encontrar material (geralmente documentos) de natureza desestruturada (geralmente texto) que satisfaça uma necessidade de informação dentro de grandes acervos (geralmente armazenados em computadores). (CITAÇAO DO LIVRO C D MANNING).