%---------------------------------------------------------------------------------
% Resumo em inglês
%---------------------------------------------------------------------------------
\setlength{\absparsep}{18pt} % ajusta o espaçamento dos parágrafos do resumo
\begin{resumo}[Abstract]
    \begin{otherlanguage*}{english}
        In the information age currently going on, the world generates a huge amount of textual data on a daily basis, and Text Mining aims to turn this data into useful information, into knowledge, and that has applications even in the forensic area.
        The study field of Information Retrieval contributes to the development of Text Mining in preprocessing, however it can also used in the creation of attributes for classifiers.
        
        From a literature review of the Information Retrieval and Text Mining areas, we suggest to create attributes using the BM25 ranking function, using existing storage and indexing tools to calculate it.
        A methodology for analyzing the performance of the suggested new attributes is established, and the measurement of classifier performance by consolidated measures in the Text Mining literature is proposed.
        	
    	\vspace{\onelineskip}
    
    	\noindent
    	\textbf{Key-words}: \textit{Text Mining}, \textit{Information Retrieval}, \textit{feature creation}, \textit{performance evaluation}, \textit{feature engineering}.

    \end{otherlanguage*}
\end{resumo}