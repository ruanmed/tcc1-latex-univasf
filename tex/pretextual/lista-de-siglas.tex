%---------------------------------------------------------------------------------
% Insere lista de abreviaturas e siglas
%---------------------------------------------------------------------------------
%   -   Use o comando \nomenclature em qualquer lugar do seu texto para definir as siglas
%       Siga o exemplo abaixo: \nomenclature{SIGLA}{DefiniçÃo da sigla}
%   \nomenclature{Fig.}{Figura TTTTT}
\begin{KeepFromToc} 
    \renewcommand{\nomname}{\listadesiglasname}
    \pdfbookmark[0]{\nomname}{las}
    \printnomenclature[2.0cm]
    \cleardoublepage
\end{KeepFromToc}

% TESTE DE SIGLAS
\nomenclature{RI}{Recuperação de Informação}

\nomenclature{MT}{Mineração de Texto}

\nomenclature{BD}{Banco de Dados}

\nomenclature{PRP}{Probability Ranking Principle (princípio de ranqueamento probabilístico)}

\nomenclature{BIM}{\textit{Binary Independence Model} (modelo de independência binária)}

\nomenclature{KDD}{\textit{Knowledge Discovery in Data} (descoberta de conhecimento em dados)}

\nomenclature{SGBD}{Sistemas Gerenciadores de Banco de Dados}

\nomenclature{PAN}{Organização que se originou do \textit{International Workshop on Plagiarism Analysis, Authorship Identification, and Near-Duplicate Detection}}

\nomenclature{CLEF}{\textit{Conference and Labs for the Evaluation Forum}}

\nomenclature{API}{\textit{Application Programming Interface} (interface de programação de aplicação)}

\nomenclature{JSON}{\textit{JavaScript Object Notation} (notação para objetos JavaScript}

\nomenclature{HTML}{\textit{HyperText Markup Language} (linguagem de marcação de hipertexto)}

\nomenclature{TREC}{\textit{Text REtrieval Conference format} (formato da Conferência de Recuperação de Texto)}

\nomenclature{AQL}{\textit{ArangoDB Query Language}}

\nomenclature{NoSQL}{\textit{Not Only SQL} ou \textit{no SQL} (não apenas SQL ou sem SQL)}

\nomenclature{SQL}{\textit{Structured Query Language} (linguagem de consulta estruturada)}

\nomenclature{PDF}{\textit{Portable Document Format} (formato de documento portátil)}

\nomenclature{XML}{\textit{Extensive Markup Language} (formato de documento portátil)}

\nomenclature{W3C}{\textit{World Wide Web Consortium}}

\nomenclature{cons.}{consulta}

\nomenclature{pos.}{posição}

\nomenclature{pont.}{pontuação}

%   -   Modo manual de fazer a lista de siglas
% \begin{siglas}
% 	\item[LI]       Lorem Ipsum
%     \item[LII]		Lorem Ipsum Ipsum
	    
% \end{siglas}


