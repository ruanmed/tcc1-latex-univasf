%--------------------------------------------------------------------------------
% Resumo em português
%--------------------------------------------------------------------------------
\setlength{\absparsep}{18pt} % ajusta o espaçamento dos parágrafos do resumo
\begin{resumo}
    
    O mundo informatizado gera uma quantidade gigantesca de dados textuais diariamente, e a Mineração de Texto objetiva transformar esses dados em informações úteis, em conhecimento, tendo aplicação inclusive na área forense.
    A área de Recuperação de Informação contribui para o desenvolvimento da Mineração de Texto no pré-processamento, porém sua utilização direta na criação de atributos não é usual, sendo proposto pela primeira vez por \citeonline{WEREN_CLEF_2014}.
    
    Após uma revisão bibliográfica das áreas de Recuperação de Informação e de Mineração de Texto, o autor sugere uma metodologia para criação de atributos utilizando a função de ranqueamento BM25 e utilizando ferramentas de armazenamento e indexação já existentes para o cálculo.
    Os atributos são criados de modo similar ao feito por \citeonline{WEREN_MESTRADO_2014}.
    Nessa metodologia a análise do desempenho dos novos atributos sugeridos é feita por meio da mensuração do desempenho de classificador por medidas consolidadas na literatura de Mineração de Texto, como acurácia e $F_1$-score.

    Os resultados obtidos a partir da reprodução da metodologia sugerida são apresentados em duas partes. 
    A primeira parte é a de desempenho das ferramentas de armazenamento e indexação que indica o Zettair como a ferramenta com melhor desempenho para indexação de documentos, e o Elasticsearch como o melhor SGBD tanto no quesito desempenho de indexação quanto de consulta.

    A segunda parte dos resultados mostra que os atributos sugeridos, derivados de Recuperação de Informação, aumentam o desempenho de classificadores aplicados a um corpus pequeno com documentos longos, 645 artigos.
    Esses mesmos atributos adicionados em outro classificador aplicado em um corpus grande com documentos pequenos, 300 mil \textit{tweets}, causa perda de desempenho.
    
    \vspace{\onelineskip}
    
	\noindent
    \textbf{Palavras-chave}: Mineração de Texto, Recuperação de Informação, criação de atributos, avaliação de desempenho, engenharia de atributos.
    
\end{resumo}