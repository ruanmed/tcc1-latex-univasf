%--------------------------------------------------------------------------------
% Resumo em português
%--------------------------------------------------------------------------------
\setlength{\absparsep}{18pt} % ajusta o espaçamento dos parágrafos do resumo
\begin{resumo}
    
    O mundo informatizado gera uma quantidade gigantesca de dados textuais diariamente, e a Mineração de Texto objetiva transformar esses dados em informações úteis, em conhecimento, tendo aplicação inclusive na área forense.
    A área de Recuperação de Informação contribui para o desenvolvimento da Mineração de Texto no pré-processamento, no entanto também pode ser incluída na criação de atributos para os classificadores.
    
    A partir de uma revisão bibliográfica das áreas de Recuperação de Informação e de Mineração de Texto, sugere-se a criação de atributos utilizando a função de ranqueamento BM25, utilizando ferramentas de armazenamento e indexação já existentes para o cálculo.
    É estabelecida uma metodologia para análise do desempenho dos novos atributos sugeridos, sendo proposto a mensuração do desempenho de classificador por medidas consolidadas na literatura de Mineração de Texto.
    
    \vspace{\onelineskip}
    
	\noindent
    \textbf{Palavras-chave}: Mineração de Texto, Recuperação de Informação, criação de atributos, avaliação de desempenho, engenharia de atributos.
    
\end{resumo}