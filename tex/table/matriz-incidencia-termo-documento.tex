\begin{table}[H]
    \centering
    \caption{Matriz de incidência de termo-documento do livro \textit{Shakespeare’s Collected Works}. Cada elemento (i, j) da matriz é 1 se a peça de teatro na coluna j contém a palavra na linha i, caso contrário o elemento é 0.}
    \begin{tabular}{|c|c|c|c|c|c|c|c}
        \hline
        \diagbox{Palavra}{
            \raisebox{-1.27cm}{
                \rotatebox{90}{
                    \parbox{1.6cm}{\centering Peça \\ de teatro}
                }
            }
        } 
        & \makecell{Antony \\ and \\ Cleopatra} 
        & \makecell{Julius \\ Caesar} 
        & The Tempest 
        & Hamlet 
        & Othello 
        & Macbeth 
        & ... 
        \\ \hline
        Antony     & 1 & 1 & 0 & 0 & 0 & 1 & \\
        Brutus     & 1 & 1 & 0 & 1 & 0 & 0 & \\
        Caeasar    & 1 & 1 & 0 & 1 & 1 & 1 & \\
        Calpurnia  & 0 & 1 & 0 & 0 & 0 & 0 & \\
        Cleopatra  & 1 & 0 & 0 & 0 & 0 & 0 & \\
        mercy      & 1 & 0 & 1 & 1 & 1 & 1 & \\
        worser     & 1 & 0 & 1 & 1 & 1 & 0 & \\
        ...        & & & & & & & 
    \end{tabular}
    \legend{\ABNTEXfontereduzida \textbf{Fonte:} Tabela adaptada de \citeonline[p.~4]{Manning2008IIR}.}
    \label{tab:matriz-incidência-termo-documento}
\end{table}