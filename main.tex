\documentclass[
	12pt,				% tamanho da fonte
	openany,			% capítulos começam em pág ímpar (insere página vazia caso preciso)
	oneside, 			% oneside - twoside
	a4paper,			% tamanho do papel.
	chapter=TITLE,		% títulos de capítulos convertidos em letras maiúsculas
	section=TITLE,		% títulos de seções convertidos em letras maiúsculas
	sumario=tradicional,	
	%subsection=TITLE,	% títulos de subseções convertidos em letras maiúsculas
	%subsubsection=TITLE,% títulos de subsubseções convertidos em letras maiúsculas
	english,			% idioma adicional para hifenização
	brazil,				% o último idioma é o principal do documento
	inline,             % adicionado para o enumitem
	shortlabels,        % adicionado para o enumitem
	]{abntex2}

% ---------------------------------------------------------------------------
% Inclui os comandos do projeto
% ---------------------------------------------------------------------------
% -----------------------------------------------------------------------------
% Pacotes fundamentais
% -----------------------------------------------------------------------------
\usepackage{xcolor}
\newcommand\myworries[1]{\textcolor{red}{[#1]}}

% Escolhendo a fonte
% Consultar catálogo de fontes: http://www.tug.dk/FontCatalogue/
% \usepackage{lmodern}        % Usa a fonte Latin Modern (Serifada, tipo Times New Roman
\usepackage{fourier}        % Usa a fonte Utopia Regular with Fourier
% \usepackage[T1]{fontenc}    % ^^^^

% \usepackage{helvet}         % Usa a fonte Helvetica (Tipo Arial)		
% \renewcommand{\familydefault}{\sfdefault}   % tira o serifado

\usepackage[T1]{fontenc}		% Selecao de codigos de fonte.
\usepackage[utf8]{inputenc}		% Codificacao do documento (conversão automática dos acentos)
\usepackage{indentfirst}		% Indenta o primeiro parágrafo de cada seção.
\usepackage{color}				% Controle das cores
\usepackage{tikz}				% Inclusão de gráficos
\usepackage{graphicx}			% Inclusão de gráficos
\usepackage{microtype} 			% para melhorias de justificação
% -----------------------------------------------------------------------------
% Pacotes adicionais, usados no anexo do modelo de folha de identificação
% -----------------------------------------------------------------------------
\usepackage{multicol}
\usepackage{multirow}
% -----------------------------------------------------------------------------
% Pacotes adicionais, usados apenas no âmbito do Modelo Canônico do abnteX2
% -----------------------------------------------------------------------------
\usepackage{lipsum}				% para geração de dummy text
% -----------------------------------------------------------------------------
% Pacotes de citações
% -----------------------------------------------------------------------------
\usepackage[brazilian,hyperpageref]{backref}	 % Paginas com as citações na bibliografia
\usepackage[alf,abnt-etal-list=3,abnt-etal-cite=3, abnt-emphasize=bf]{abntex2cite}	% Citações padrão ABNT
\usepackage{pdflscape}
\usepackage{footnote}
\usepackage{pdfpages}
\usepackage[skip=0pt]{caption} % Configurado para não ter espaço extra após o rótulo da figura
% \setlength{\abovecaptionskip}{0pt plus 0pt minus 0pt} % https://tex.stackexchange.com/questions/45990/how-can-i-modify-vertical-space-between-figure-and-caption
% \setlength{\belowcaptionskip}{0pt plus 0pt minus 0pt}
% \usepackage{caption}

% -----------------------------------------------------------------------------
% Pacotes adicionados por @leolleocomp
% ----------------------------------------------------------------------------- 
\usepackage{booktabs}
\usepackage{adjustbox}
\usepackage{subcaption}
\usepackage[labelfont=bf]{caption}
\usepackage{gensymb}
\usepackage{amsmath}
\usepackage{array}
\usepackage{float}
\usepackage{xcolor,colortbl}
\usepackage{longtable}
\usepackage{scalefnt}
\usepackage{listings}			% inserir codigo fonte


% -----------------------------------------------------------------------------
% Pacotes adicionados por @Gabrielr2508
% ----------------------------------------------------------------------------- 
\usepackage{hyperref}

% Para forçar quebra de linhas em urls muito longos
% https://tex.stackexchange.com/questions/3033/forcing-linebreaks-in-url
% \RequirePackage[hyphens]{url}
% \PassOptionsToPackage{hyphens}{url}\usepackage{hyperref}

% \usepackage{hyperref}

\usepackage{tocloft}
% -- permite a adição de células especiais em tabelas
\newcommand{\specialcell}[2][c]{%
  \begin{tabular}[#1]{@{}c@{}}#2\end{tabular}}

\newcounter{equationset}
\newcommand{\equationset}[1]{% \equationset{<caption>}
  \refstepcounter{equationset}% Step counter
  \noindent\makebox[\linewidth]{Equação ~\theequationset: #1}
 }

%--------------------------------------------------------------------------------
% Adequação dos títulos dos capitulos, seções, subseções às normas da Univasf
% Added by @Gabrielr2508
%--------------------------------------------------------------------------------
\renewcommand{\ABNTEXchapterfont}{\fontseries{b}}
\renewcommand{\ABNTEXchapterfontsize}{\normalsize}

\renewcommand{\ABNTEXsectionfont}{\fontseries{m}}
\renewcommand{\ABNTEXsectionfontsize}{\normalsize}

\renewcommand{\ABNTEXsubsectionfont}{\fontseries{b}}
\renewcommand{\ABNTEXsubsectionfontsize}{\normalsize}

\renewcommand{\ABNTEXsubsubsectionfont}{\fontseries{m}}
\renewcommand{\ABNTEXsubsubsectionfontsize}{\normalsize}

%--------------------------------------------------------------------------------
% CONFIGURAÇÕES DE PACOTES
% Configurações do pacote backref
%--------------------------------------------------------------------------------
% Usado sem a opção hyperpageref de backref
\renewcommand{\backrefpagesname}{Citado na(s) página(s):~}
% Texto padrão antes do número das páginas
\renewcommand{\backref}{}
% Define os textos da citação
\renewcommand*{\backrefalt}[4]{
	\ifcase #1 %
		%Nenhuma citação no texto.%
	\or
		Citado na página #2.%
	\else
		Citado #1 vezes nas páginas #2.%
	\fi}%

%--------------------------------------------------------------------------------
% Configurações de aparência do PDF final
%--------------------------------------------------------------------------------
% alterando o aspecto da cor azul
\definecolor{blue}{RGB}{41,5,195}

% informações do PDF
\makeatletter
\hypersetup{
     	%pagebackref=true,
		pdftitle={\@title},
		pdfauthor={\@author},
    	pdfsubject={\imprimirpreambulo},
	    pdfcreator={LaTeX with abnTeX2},
		pdfkeywords={abnt}{latex}{abntex}{abntex2}{relatório técnico},
		colorlinks=true,			% false: boxed links; true: colored links
    	linkcolor=black,				% color of internal links
    	citecolor=black,				% color of links to bibliography
    	filecolor=black,			% color of file links
		urlcolor=black,
		bookmarksdepth=4
}
\makeatother
% ---

% ---
% Espaçamentos entre linhas e parágrafos
% ---

% O tamanho do parágrafo é dado por:
\setlength{\parindent}{1.3cm}

% Controle do espaçamento entre um parágrafo e outro:
\setlength{\parskip}{0.2cm}  % tente também \onelineskip

% Controle do espaçamento em listas de itens
\setlist[itemize]{leftmargin=1.7cm}
\setlist[enumerate]{leftmargin=1.7cm}

%--------------------------------------------------------------------------------
% compila o índice
%--------------------------------------------------------------------------------
\makeindex
% ---

%--------------------------------------------------------------------------------
% Comando para inserir imagens de forma simples
%--------------------------------------------------------------------------------
\newcommand{\imagem}[4]
{%			\imagem{x.x}{nomeimg}{titulo}{fonte}
	\begin{figure}[!htb]
		\caption{\label{img:#2}#3}
		\begin{center}
			\includegraphics[scale=#1]{img/#2}
		\end{center}
        \legend{\textbf{Fonte:} #4}
	\end{figure}
}%

%--------------------------------------------------------------------------------
% Creio que esses comandos sejam para desenhar algo, aguardando explicações de @leolleocomp
%--------------------------------------------------------------------------------
\newcommand{\xx} {$\bigotimes$}
\newcommand{\oo} {$\bigcirc$}

%--------------------------------------------------------------------------------
% Biblioteca para códigos-fonte
%--------------------------------------------------------------------------------
\usepackage[newfloat=true]{minted}

%--------------------------------------------------------------------------------
% Caixas batutas - by @leolleocomp
%--------------------------------------------------------------------------------
\usepackage[most]{tcolorbox}
\tcbuselibrary{breakable}

\tcbuselibrary{minted}
\tcbset{listing engine=minted}

\definecolor{bg}{rgb}{0.95,0.95,0.95}

\SetupFloatingEnvironment{listing}{name=Código, listname=Lista de códigos}

%--------------------------------------------------------------------------------
% configuração do contador dos códigos-fonte - by @leolleocomp
% assim como as figuras, começa em 1
\newcounter{sourcecode}
%--------------------------------------------------------------------------------
%--------------------------------------------------------------------------------
% @leolleocomp
% stackoverflow code
% peguei da resposta abaixo
% https://stackoverflow.com/questions/24086366/change-latex-minted-listings-numbering-to-include-current-section?answertab=votes#tab-top
%--------------------------------------------------------------------------------
\makeatletter
\renewcommand*{\thelisting}{\thesourcecode}
\makeatother

%--------------------------------------------------------------------------------
% Peçam explicações a @leolleo
% WHO DID THIS?
%--------------------------------------------------------------------------------
\newcommand{\Ididthis}{
%	\legend{\textbf{Fonte:} O autor (\the\year).}
\legend{\textbf{Fonte:} O autor}
}

\newcommand{\Otherguydidthis}[1]{
	\legend{\textbf{Fonte:} \citeonline{#1}.}
}

%--------------------------------------------------------------------------------
% Comando para inserir códigos - by @leolleocomp
%--------------------------------------------------------------------------------
\newcommand{\sourcecode}[4]{
\begin{listing}[H]
	\refstepcounter{sourcecode}
	\caption{#1}
	\label{cmd:#2}
	\inputminted[linenos, bgcolor=bg, tabsize=4,breaklines]{#3}{codes/#4}
	\Ididthis
\end{listing}
}


% -----------------------------------------------------------------------------
% Pacotes adicionados por @ruanmed
% ----------------------------------------------------------------------------- 
\usepackage[binary-units=true]{siunitx}     %   Pacote de unidades do SI com unidades binárias


%--------------------------------------------------------------------------------
% Comando para inserir códigos - by @ruanmed
%--------------------------------------------------------------------------------
% \usepackage{caption}

% \newenvironment{code}{\captionsetup{type=listing}}{}
% \SetupFloatingEnvironment{listing}{name=Código}

% \newcommand{\sourcecodenolist}[4]{
% 	\begin{code}
% 	    \refstepcounter{sourcecode}
%         \captionof{listing}{#1 }
%         \label{code:#2}
%         \inputminted[linenos, bgcolor=bg, tabsize=4,breaklines]{#3}{codes/#4}
%         \Wedidthis
%     \end{code}
% }

% \newcommand{\sourcecodenolist}[4]{
% 	\refstepcounter{sourcecode}
% 	\captionof{listing}{#1 \label{cmd:#2}}
% 	\inputminted[linenos, bgcolor=bg, tabsize=4,breaklines]{#3}{codes/#4}
% 	\Wedidthis
% }

% \newcommand{\sourcecodenolist}[4]{
% 	\refstepcounter{sourcecode}
% 	\captionof{listing}{#1 \label{cmd:#2}}
% 	\inputminted[linenos, bgcolor=bg, tabsize=4,breaklines]{#3}{codes/#4}
% 	\Wedidthis
% }


\newcommand{\sourcecodeinline}[2]{
	\mintinline[linenos, bgcolor=bg, tabsize=4,breaklines]{#1}{#2}
}

%--------------------------------------------------------------------------------
% Outros pacotes adicionais para circuitos - by @ruanmed
%--------------------------------------------------------------------------------
\RequirePackage{tikz}
\usepackage{tikz}
\usepackage[siunitx]{circuitikz}			% para habilitar o desenho de circuitos
\usetikzlibrary{babel}
% ---

%  https://tex.stackexchange.com/questions/324951/how-can-i-draw-a-vector-diagram-that-illustrates-polar-and-rectangular-coordinat
\usetikzlibrary{angles, arrows.meta, quotes}


% Pacote para usar modulo de vetor || v || com \norm - https://tex.stackexchange.com/questions/43008/absolute-value-symbols
\usepackage{commath}

%--------------------------------------------------------------------------------
% Outros pacotes adicionais para Siglas - by @ruanmed
%--------------------------------------------------------------------------------
\usepackage{nomencl}
	\makenomenclature

% Pacote para fazer listas enumeradas na mesma linha
% http://www.texnia.com/archive/enumitem.pdf
% % \usepackage{enumerate}
% \usepackage[inline, shortlabels]{enumitem}

\newlist{inlinelist}{enumerate*}{1}
\setlist*[inlinelist,1]{%
  label=(\alph*),
}

% Pacote para inserir divisória diagonal em tabela
\usepackage{diagbox}

% Pacote para auxiliar em tabelas criando novas células
\usepackage{makecell}

\usepackage{pbox,ragged2e}



% Gráficos
\usepackage{tikz}
\usepackage{pgfplots}
\pgfplotsset{compat=1.16}

% ---------------------------------------------------------------------------
% IDENTIFICAÇÃO
% ---------------------------------------------------------------------------
\titulo{Avaliação de técnicas de Recuperação de Informação como variáveis em Mineração de Textos}
\autor{RUAN DE MEDEIROS BAHIA}
\local{JUAZEIRO - BA}
\orientador{Prof. Dr. Rosalvo Oliveira Neto}
%\coorientador{M. Sc. Ciclano Fulado de Tal}
\instituicao{
    UNIVERSIDADE FEDERAL DO VALE DO SÃO FRANCISCO
    	\\
    CURSO DE GRADUAÇÃO EM ENGENHARIA DE COMPUTAÇÃO}
\tipotrabalho{Trabalho de Conclusão de Curso}
\preambulo{Trabalho apresentado à Universidade Federal do Vale do São Francisco - Univasf, Campus Juazeiro, como requisito da obtenção do título de Bacharel em Engenharia de Computação.}


% -----------------------------------------------------------------------------
% CONFIGURACAO DO SUMARIO - by @Gabrielr2508
% Precisa estar aqui, por isso não foi para o commands.tex, não descobrimos o motivo, %caso saiba, por favor, faça um pull request! :D
% -----------------------------------------------------------------------------
% Secao primaria (Chapter) Caixa alta, Negrito, tamanho 12
\makeatletter
\renewcommand*{\l@chapter}[2]{%
  \l@chapapp{\uppercase{#1}}{#2}{\cftchaptername}}
\makeatother
% Secao secundaria (Section) Caixa baixa, Negrito, tamanho 12
\renewcommand{\cftsectionfont}{\uppercase} %ponha \rmfamily se quiser serifadas...

% Secao terciaria (Subsection) Caixa baixa, negrito, tamanho 12
\renewcommand{\cftsubsectionfont}{\bfseries}

% Secao quaternaria (Subsubsection) Caixa baixa, tamanho 12
\renewcommand{\cftsubsubsectionfont}{\normalfont}

% Seção quinaria (subsubsubsection) Caixa baixa, sem negrito, tamanho 12
\renewcommand{\cftparagraphfont}{\normalfont\itshape}

% -----------------------------------------------------------------------------
% Início do TCC 
% -----------------------------------------------------------------------------
\begin{document}
\selectlanguage{brazil}
	\frenchspacing % Retira espaço extra obsoleto entre as frases.
	
	\pretextual
		%--------------------------------------------------------------------------------
% Capa
%--------------------------------------------------------------------------------
%--------------------------------------------------------------------------------
% Constrói a capa com base na seção de identificação do main.tex
%--------------------------------------------------------------------------------
\begin{capa}
    \setlength{\belowcaptionskip}{0pt}
    \setlength{\abovecaptionskip}{0pt}
    \setlength{\intextsep}{-18pt}
        \begin{figure}[h]
    		\begin{center}
    		    % \includegraphics[scale=1.0]{img/LOGO_UNIVASF_big.pdf}
    		    \includegraphics[scale=0.5]{img/marca-univasf-completa-sem-fundo.png}
    		    % \includegraphics[scale=0.5]{img/marca-univasf-simplificada-sem-fundo.png}
    		\end{center}
    	\end{figure}
        
        %\includegraphics[scale=0.6]{img/univasf.jpg}
        \center
    	{\ABNTEXchapterfont\large\imprimirinstituicao}
	
		% \MakeUppercase{Texto}
		
    	\vspace*{2cm}
    	    {\imprimirautor}
    	\vspace*{2cm}
        \begin{center}
    		\ABNTEXchapterfont\bfseries\large\imprimirtitulo
        \end{center}
    	\vfill
        
    	\ABNTEXchapterfont\bfseries\large\imprimirlocal\\ 
    	\the\year
    	
    	\vspace*{1cm}
\end{capa}

%--------------------------------------------------------------------------------
% Folha de rosto 
%--------------------------------------------------------------------------------
%--------------------------------------------------------------------------------
% Constrói a folha de rosto com base na seção de identificação do main.tex
%--------------------------------------------------------------------------------
\begin{folhaderosto}
    \center
    	{\ABNTEXchapterfont\large\imprimirinstituicao}
    	
		\vspace*{2cm}
    	    {\imprimirautor}
    	\vspace*{2cm}
		\vspace*{\fill}

		{\ABNTEXchapterfont\bfseries\large\imprimirtitulo}
		\vspace*{\fill}

		{\hspace{.45\textwidth}
		\begin{minipage}{.5\textwidth}
			\SingleSpacing
			\imprimirpreambulo 

			{\imprimirorientadorRotulo~\imprimirorientador\par}
			{\imprimircoorientadorRotulo~\imprimircoorientador\par}

		\end{minipage}%
		\vspace*{\fill}}%
		\vspace*{\fill}
			\ABNTEXchapterfont\bfseries\large\imprimirlocal\\ 
			\the\year
		\vspace*{1cm}
\end{folhaderosto}

%--------------------------------------------------------------------------------
% Constrói a ficha catalográfia com base na seção de identificação do main.tex
% Está comentado porque no final das contas a biblioteca do seu campus que gera a 
% numeração, você pode adicionar os numeros aqui, ou anexar o pdf gerado por eles
% ao documento.
%--------------------------------------------------------------------------------
%--------------------------------------------------------------------------------
% Constrói a ficha catalográfia com base na seção de identificação do main.tex
%--------------------------------------------------------------------------------
\begin{fichacatalografica}
	\vspace*{\fill}					% Posição vertical
	\hrule							% Linha horizontal
	\begin{center}					% Minipage Centralizado
	\begin{minipage}[c]{12.5cm}		% Largura
    	\imprimirautor
    
    	\hspace{0.5cm} \imprimirtitulo  / \imprimirautor. --
    	\imprimirlocal, \the\year-
    
    	\hspace{0.5cm} xx p. : il. (algumas color.) ; 30 cm.\\
    
    	\hspace{0.5cm} \imprimirorientadorRotulo~\imprimirorientador\\
    
    	\hspace{0.5cm}
    	\parbox[t]{\textwidth}{\imprimirtipotrabalho~--~\imprimirinstituicao,
    	\the\year.}\\
    
    	\hspace{0.5cm}
    		1. Palavra-chave1.
    		2. Palavra-chave2.
    		I. Rosalvo Ferreira de Oliveira Neto. % I. Orientador.
    		II. Universidade Federal do Vale do São Francisco% II. Universidade xxx.
    		III. Faculdade de xxx.
    		IV. Título\\
    
    	\hspace{8.75cm} CDU 02:141:005.7\\
	\end{minipage}
	\end{center}
	\hrule
\end{fichacatalografica}

%--------------------------------------------------------------------------------
% Anexando a ficha catalogáfica e a folha de aprovação 
%--------------------------------------------------------------------------------
% \includepdf[pages=-]{anexos/ficha.pdf}

% \includepdf[pages=-]{anexos/aprovacao.pdf}

\setlength{\ABNTEXsignwidth}{12cm}

%--------------------------------------------------------------------------------
% Está comentado pelo mesmo motivo da ficha catalográfica 
%--------------------------------------------------------------------------------
%--------------------------------------------------------------------------------
% Insere a folha de aprovação 
%--------------------------------------------------------------------------------
\begin{folhadeaprovacao}
	\begin{center}
		{\ABNTEXchapterfont\bfseries\large\imprimirinstituicao}
		\vspace*{\fill}

		{\ABNTEXchapterfont\bfseries\large FOLHA DE APROVAÇÃO}
		\vspace*{\fill}

		{\ABNTEXchapterfont\bfseries\large\imprimirautor}

		\vspace*{\fill}\vspace*{\fill}
		{\ABNTEXchapterfont\bfseries\large\imprimirtitulo}
		\vspace*{\fill}

		{\hspace{.45\textwidth}
		\begin{minipage}{.5\textwidth}
			\SingleSpacing
			\ABNTEXchapterfont\imprimirpreambulo \\ \\

			% {\ABNTEXchapterfont\imprimirorientadorRotulo~\imprimirorientador\par}
			% {\ABNTEXchapterfont\imprimircoorientadorRotulo~\imprimircoorientador\par}

		\end{minipage}%
		\vspace*{\fill}}
	\end{center}

	\vspace*{\fill}	
	
	\begin{center}
		% \ABNTEXchapterfont\large Aprovado em: \_\_\_\_ de \_\_\_\_\_\_\_\_ de 2020.
		\ABNTEXchapterfont\large Aprovado em: 31 de janeiro de 2020.
	\end{center}

	\vspace*{\fill}
	
	\begin{center}
			 \ABNTEXchapterfont\bfseries\large Banca Examinadora
	\end{center}
		
	\ABNTEXchapterfont\assinatura{Rosalvo Ferreira de Oliveira Neto, Doutor, Universidade Federal do Vale do São Francisco}	
	\ABNTEXchapterfont\assinatura{Juracy Emanuel Magalhães da Franca, Mestre, Universidade Federal do Vale do São Francisco}
	\ABNTEXchapterfont\assinatura{Marcus Vinícius Midena Ramos, Doutor, Universidade Federal do Vale do São Francisco}
	\vspace*{\fill}

	 
\end{folhadeaprovacao}

%--------------------------------------------------------------------------------
% Insere a epígrafe
%--------------------------------------------------------------------------------
% %--------------------------------------------------------------------------------
% Insere a epígrafe
%--------------------------------------------------------------------------------
\newpage
\vspace*{\fill}
\begin{flushright}
		\textit{Lorem Ipsum...}
\end{flushright}

%--------------------------------------------------------------------------------
% Seção de agradecimentos
%--------------------------------------------------------------------------------
%--------------------------------------------------------------------------------
% Seção de agradecimentos
%--------------------------------------------------------------------------------
\begin{agradecimentos}
	

    O conhecimento humano não é desenvolvido individualmente, inúmeros indivíduos colaboraram para a realização deste trabalho.
    Muitos livros, artigos, trabalhos, e linhas de código escritas por pessoas de vários países deixaram um pouco de conhecimento comigo.
    Diversas pessoas contribuíram para minha formação como ser humano, família, amigos, professores, colegas, conhecidos e desconhecidos, e a essas pessoas eu sou grato.

    Dedico este trabalho ao meu pai, Antonio, pois se segui esse caminho com certeza foi devido à influência dele.

\end{agradecimentos}

%--------------------------------------------------------------------------------
% Insere a segunda epígrafe
%--------------------------------------------------------------------------------
% %--------------------------------------------------------------------------------
% Insere a segunda epígrafe
%--------------------------------------------------------------------------------
\begin{epigrafe}
    \vspace*{\fill}
	\begin{flushright}
		Se pude enxergar a tão grande distância, foi subindo nos ombros de gigantes.\\
		 \vspace{\baselineskip}
		\textbf{Isaac Newton}\\
		\textbf{Carta à Robert Hooke, 1676}
	\end{flushright}
\end{epigrafe}

%--------------------------------------------------------------------------------
% Seção de resumos
%--------------------------------------------------------------------------------
% resumo em português
%--------------------------------------------------------------------------------
% Resumo em português
%--------------------------------------------------------------------------------
\setlength{\absparsep}{18pt} % ajusta o espaçamento dos parágrafos do resumo
\begin{resumo}
    
    O mundo informatizado gera uma quantidade gigantesca de dados textuais diariamente, e a Mineração de Texto objetiva transformar esses dados em informações úteis, em conhecimento, tendo aplicação inclusive na área forense.
    A área de Recuperação de Informação contribui para o desenvolvimento da Mineração de Texto no pré-processamento, porém sua utilização direta na criação de atributos não é usual, sendo proposto pela primeira vez por \citeonline{WEREN_CLEF_2014}.
    
    Após uma revisão bibliográfica das áreas de Recuperação de Informação e de Mineração de Texto, o autor sugere uma metodologia para criação de atributos utilizando a função de ranqueamento BM25 e utilizando ferramentas de armazenamento e indexação já existentes para o cálculo.
    Os atributos são criados de modo similar ao feito por \citeonline{WEREN_MESTRADO_2014}.
    Nessa metodologia a análise do desempenho dos novos atributos sugeridos é feita por meio da mensuração do desempenho de classificador por medidas consolidadas na literatura de Mineração de Texto, como acurácia e $F_1$-score.

    Os resultados obtidos a partir da reprodução da metodologia sugerida são apresentados em duas partes. 
    A primeira parte é a de desempenho das ferramentas de armazenamento e indexação que indica o Zettair como a ferramenta com melhor desempenho para indexação de documentos, e o Elasticsearch como o melhor SGBD tanto no quesito desempenho de indexação quanto de consulta.

    A segunda parte dos resultados mostra que os atributos sugeridos, derivados de Recuperação de Informação, aumentam o desempenho de classificadores aplicados a um corpus pequeno com documentos longos, 645 artigos.
    Esses mesmos atributos adicionados em outro classificador aplicado em um corpus grande com documentos pequenos, 300 mil \textit{tweets}, causa perda de desempenho.
    
    \vspace{\onelineskip}
    
	\noindent
    \textbf{Palavras-chave}: Mineração de Texto, Recuperação de Informação, criação de atributos, avaliação de desempenho, engenharia de atributos.
    
\end{resumo}

%---------------------------------------------------------------------------------
% resumo em inglês
%---------------------------------------------------------------------------------
% Resumo em inglês
%---------------------------------------------------------------------------------
\setlength{\absparsep}{18pt} % ajusta o espaçamento dos parágrafos do resumo
\begin{resumo}[Abstract]
    \begin{otherlanguage*}{english}
        In the information age currently going on, the world generates a huge amount of textual data on a daily basis, and Text Mining aims to turn this data into useful information, knowledge, and that has applications even in the forensic area.
        The study field of Information Retrieval contributes to the development of Text Mining in preprocessing, however it can also used in the creation of attributes for classifiers.
        
        From a literature review of the study fields of  Information Retrieval and Text Mining, we suggest to create attributes using the BM25 ranking function, similar to attributes created by \citeonline{WEREN_MESTRADO_2014}, and using existing storage and indexing tools to calculate them.
        A methodology for analyzing the performance of the suggested new attributes is established, and the measurement of classifier performance by consolidated measures in the Text Mining literature is proposed, such measurements are accuracy and $F_1$-score.

        The results obtained by reproducing the suggested methodology demonstrate Zettair as the tool with the best perfomance for document indexing, and Elasticsearch as the best DBMS both in terms of indexing and query performance.

        The suggested attributes, derived from Information Retrieval, exibith performance gains in classifiers of a small corpus with long documents, 645 articles.
        These same attributes applied in another classifier for a large corpus with small documents, 300 thousand \ textit {tweets}, cause this classifier to lose performance.
   
    	\vspace{\onelineskip}
    
    	\noindent
    	\textbf{Key-words}: \textit{Text Mining}, \textit{Information Retrieval}, \textit{feature creation}, \textit{performance evaluation}, \textit{feature engineering}.

    \end{otherlanguage*}
\end{resumo}


%---------------------------------------------------------------------------------
% Insere lista de ilustrações
%---------------------------------------------------------------------------------
\begin{KeepFromToc}  % Este comando evita que todas as seções dentro dele de apareçam no sumário

%---------------------------------------------------------------------------------
% Insere lista de ilustrações
%---------------------------------------------------------------------------------
\pdfbookmark[0]{\listfigurename}{lof}
\listoffigures
%\addcontentsline{toc}{chapter}{Lista de Figuras}
\cleardoublepage

\end{KeepFromToc}

%---------------------------------------------------------------------------------
% Insere lista de tabelas
%---------------------------------------------------------------------------------
\begin{KeepFromToc}  % Este comando evita que todas as seções dentro dele de apareçam no sumário

%---------------------------------------------------------------------------------
% Insere lista de tabelas
%---------------------------------------------------------------------------------
\pdfbookmark[0]{\listtablename}{lot}
\listoftables
\cleardoublepage

\end{KeepFromToc}
%---------------------------------------------------------------------------------
% Insere lista de códigos - by @leolleocomp
%---------------------------------------------------------------------------------
\begin{KeepFromToc}  % Este comando evita que todas as seções dentro dele de apareçam no sumário
%---------------------------------------------------------------------------------
% Ajusta lista de código - alterar de figures para códigos - by @Gabrielr2508
%---------------------------------------------------------------------------------
\makeatletter
    \let\l@listing\l@figure
    \def\newfloat@listoflisting@hook{\let\figurename\listingname}
\makeatother

%---------------------------------------------------------------------------------
% Insere lista de códigos - by @leolleocomp
%---------------------------------------------------------------------------------
\listoflistings

\end{KeepFromToc}


%---------------------------------------------------------------------------------
% Insere lista de abreviaturas e siglas
%---------------------------------------------------------------------------------
%---------------------------------------------------------------------------------
% Insere lista de abreviaturas e siglas
%---------------------------------------------------------------------------------
%   -   Use o comando \nomenclature em qualquer lugar do seu texto para definir as siglas
%       Siga o exemplo abaixo: \nomenclature{SIGLA}{DefiniçÃo da sigla}
%   \nomenclature{Fig.}{Figura TTTTT}
\begin{KeepFromToc} 
    \renewcommand{\nomname}{\listadesiglasname}
    \pdfbookmark[0]{\nomname}{las}
    \printnomenclature[2.0cm]
    \cleardoublepage
\end{KeepFromToc}

% TESTE DE SIGLAS
\nomenclature{RI}{Recuperação de Informação}

\nomenclature{MT}{Mineração de Texto}

\nomenclature{BD}{Banco de Dados}

\nomenclature{PRP}{Probability Ranking Principle (princípio de ranqueamento probabilístico)}

\nomenclature{BIM}{\textit{Binary Independence Model} (modelo de independência binária)}

\nomenclature{KDD}{\textit{Knowledge Discovery in Data} (descoberta de conhecimento em dados)}

\nomenclature{SGBD}{Sistemas Gerenciadores de Banco de Dados}

\nomenclature{PAN}{Organização que se originou do \textit{International Workshop on Plagiarism Analysis, Authorship Identification, and Near-Duplicate Detection}}

\nomenclature{CLEF}{\textit{Conference and Labs for the Evaluation Forum}}

\nomenclature{API}{\textit{Application Programming Interface} (interface de programação de aplicação)}

\nomenclature{JSON}{\textit{JavaScript Object Notation} (notação para objetos JavaScript)}

\nomenclature{HTML}{\textit{HyperText Markup Language} (linguagem de marcação de hipertexto)}

\nomenclature{TREC}{\textit{Text REtrieval Conference format} (formato da Conferência de Recuperação de Texto)}

\nomenclature{AQL}{\textit{ArangoDB Query Language}}

\nomenclature{NoSQL}{\textit{Not Only SQL} ou \textit{no SQL} (não apenas SQL ou sem SQL)}

\nomenclature{SQL}{\textit{Structured Query Language} (linguagem de consulta estruturada)}

\nomenclature{PDF}{\textit{Portable Document Format} (formato de documento portátil)}

\nomenclature{XML}{\textit{Extensive Markup Language} (formato de documento portátil)}

\nomenclature{W3C}{\textit{World Wide Web Consortium}}

\nomenclature{cons.}{consulta}

\nomenclature{pos.}{posição}

\nomenclature{pont.}{pontuação}

\nomenclature{CNN}{\textit{Convolutional Neural Network} (Rede Neural Convolucional)}

\nomenclature{SVM}{\textit{Support Vector Machines} (Máquinas de Vetor de Suporte)}

\nomenclature{SVC}{\textit{C-Support Vector Classification} (classificação por Máquinas de Vetor de Suporte com parâmetro C)}

%   -   Modo manual de fazer a lista de siglas
% \begin{siglas}
% 	\item[LI]       Lorem Ipsum
%     \item[LII]		Lorem Ipsum Ipsum
	    
% \end{siglas}




%---------------------------------------------------------------------------------
% Insere o sumario
%---------------------------------------------------------------------------------
\input{tex/pretextual/sumario.tex}




	\textual
		\pagestyle{simple}
		%--------------------------------------------------------------------------------------
% Este arquivo contém a sua introdução, objetivos e organização do trabalho
%--------------------------------------------------------------------------------------
\chapter{Introdução} \label{ch:Introdução}
    A informatização do mundo, juntamente com a evolução dos equipamentos computacionais, vem permitindo a geração e coleta de enormes volumes de dados das mais variadas fontes, podendo-se afirmar que vivemos na era dos dados \cite[p.~1]{Han:2011:DMC:1972541}.
    Grande parte dos dados criados \textit{online} está em forma de texto (escrito em linguagem natural) e um estudo feito pela Universidade da Califórnia em Berkeley em 2003 apontou, por exemplo, que somente notícias de jornais (considerando armazenamento digital do texto dos mesmos) representavam cerca de 13,5 terabytes por ano, livros cerca de 5,5 terabytes por ano, e e-mails mais de 440 exabytes por ano \cite{lyman2003much} \cite[p.~3]{Zhai2016TDMA}.
    
    A coleta e análise da sobrecarga diária de dados se apresenta como um problema que a Mineração de Texto tenta resolver no caso de dados textuais, utilizando de técnicas de mineração de dados, aprendizado de máquina, processamento de linguagem natural, Recuperação de Informação, e gerenciamento do conhecimento \cite[p.~1]{Han:2011:DMC:1972541} \cite{Feldman:2006:TMH:1076381}.
    Técnicas de Mineração de Texto são aplicadas na classificação e clusterização de documentos, sumarização de opiniões na internet, acesso de dados biomédicos \cite[p.~4--8]{Aggarwal_MTD_2012},
    e também em tarefas de identificação de perfis de autoria\footnotemark{} \cite[p.~906]{rangel2014overview} \cite[p.~6--7]{rangel2018overview}, auxiliando em investigações forenses linguísticas \cite{Chaski_Author_2012}. 
    
    \footnotetext{A identificação de perfis de autoria consiste na extração de características do autor com base no conteúdo e estilo do texto. Essas características podem ser gênero, faixa etária, escolaridade, entre outras \cite[p.~266]{WEREN_ARTIGO_2014}.}
    % A aplicação de técnicas de Mineração de Texto nos grandes volumes de dados
    %  e é portanto uma necessidade importante,
    
% Motivação
% Importância de MT, citar exemplos práticos (encontrar criminosos, forense ) CHECK
    % motivação informal
    % existe essa área, MT, que é útil para diversas tarefas, como por exemplo classificação de grandes volumes de texto, identificação de perfil de autor e tal.
    % ela utiliza de suporte a RI para fazer suas funções, e nas tarefas de classificação são indicados atributos do dados na representação, a engenharia de atributos permite então definir bons atributos para dados textuais
    % dentre os atributos, podem ser gerados atributos oriundos de funções de RI
    % na MT cada modelo de classificação sofre uma avaliação comparativa de desempenho, o uso de diferentes atributos impacta no desempenho do classificador
    % assim, encontrar um conjunto de atributos que melhore o desempenho de classificadores de MT é uma tarefa difícil
    % encontrar meios de melhorar o desempenho dos classificadres é um dos objetivos da MT, e utilizar novos conjuntos de atributos pode trazer avanços nesse sentido
    \begin{figure}[h]
    \centering
    \includegraphics[width=0.75\textwidth]{img/figure15-1-zhai-traduzido.png}
    \caption{Tarefa de categorização de texto (considerando exemplos de treinamento disponíveis). Figura obtida e adaptada de \citeonline[p.~300]{Zhai2016TDMA}.}
    \label{fig:categorização-de-texto-zhai2016}
\end{figure}
    
    A Mineração de Texto aborda, dentre seus tipos de tarefas oriundas da mineração de dados, a tarefa de classificação nas coleções de documentos, geralmente chamada de classificação de texto ou categorização de texto \cite[p.~35]{Zhai2016TDMA}.
    Classificação é definido como o processo de designar uma, ou mais, categorias a cada objeto de texto, dentre categorias predefinidas, sendo que predominantemente é utilizado um conjunto de textos já classificado para treinamento \cite[p.~7]{Jo2018TMCIBDC} \cite[p.~299]{Zhai2016TDMA}, esse processo está exemplificado na Figura \ref{fig:categorização-de-texto-zhai2016}.
    
    No processo de classificação de texto são derivados atributos dos objetos de texto originais, um passo necessário para funcionamento do modelo de classificação originário do aprendizado de máquina.
    Diferentes conjuntos de atributos podem impactar diretamente no desempenho de um classificador \cite[p.~304--306]{Zhai2016TDMA}, o qual é tipicamente mensurado pela acurácia\footnotemark{} \cite[p.~313--314]{Zhai2016TDMA} \cite[p.~9]{Jo2018TMCIBDC}.
    
    \footnotetext{Número de previsões corretas dividido pelo número total de previsões feitas \cite[p.~313]{Zhai2016TDMA}.}
    
    A sugestão de parâmetros
    
    Dentre as métricas para 
    
    
    A Mineração de Texto utiliza de diversas técnicas desenvolvidas pela área de Recuperação de Informação no seu processamento de texto
    
    Sistemas de RI são feitos para armazenar documentos e permitirem a sua consulta por usuários, 

% Definição do Problema
% Complexidade de RI na MT.	Dizer que RI é uma área, e vamos buscar ferramentas que subsidiam o cálculo das variáveis de RI. 


% Objetivo Geral
% Objetivo Especifico
% Organização do Trabalho
% \lipsum[5-10]

    \section{Justificativa} \label{sec:Justificativa}

% \lipsum[1]


    \section{Objetivos gerais} \label{sec:Objetivos-gerais}

% \lipsum[1]
% Avaliar o desempenho de técnicas de RI como atributos em MT
% Avaliar o desempenho de ferramentas computacionais para construção (indexação) de BDs e criação dos atributos de RI.
% Avaliar o desempenho dos classificadores de MT com e sem as var de RI.

    \section{Objetivos específicos} \label{sec:Objetivos-específicos}
    
    \begin{itemize}
    	\item x1;
        \item x2;
        \item x3.
    \end{itemize}
    
    \section{Organização do trabalho} \label{sec:Organização-do-trabalho}

% \lipsum[10-12]

		%--------------------------------------------------------------------------------------
% Este arquivo contém a sua fundamentação teórica
%--------------------------------------------------------------------------------------
\chapter{Fundamentação teórica} \label{ch:FundamentaçãoTeórica}
    A utilização de funções de ranqueamento da Recuperação de Informação engloba o conhecimento acerca de fundamentos da área que serão abordados na Seção \ref{sec:RecuperaçãoInformação}, que discute desde o surgimento da área até sua evolução para os métodos probabilísticos culminando na função de ranqueamento BM25.

    Para a criação de atributos em tarefas de Mineração de Texto é necessário entender o processo de descoberta do conhecimento da Mineração de Dados, as peculiaridades para criação destes atributos, e ainda métodos de avaliação dos atributos criados. 
    Isso é feito na Seção \ref{sec:MineraçãoTexto}.

    \section{Recuperação de Informação} \label{sec:RecuperaçãoInformação}
% Falar da recuperação de informação com conceito humano

% Estrutura

% - Histórico de RI, explicando o surgimento de métodos de RI com as bibliotecas

    A busca por informação é uma necessidade humana, e uma das principais maneiras de obtê-la é consultar outras pessoas.
    No entanto, devido ao grande acúmulo de informação das sociedades, uma pessoa não pode carregar consigo todo o conhecimento do mundo.
    Assim, um modo considerado primordial de transferir esse conhecimento, que tratamos aqui como informação, é por meio de registros físicos em papel, livros e similares \cite[p.~1]{Grossman2004IRAH}.
    
    No processo de organização desses registros físicos, é notória a função dos bibliotecários de separar os vários tipos de conhecimento que os mais diversos livros podem abrigar e, portanto, os sistemas de classificação de áreas e subáreas do conhecimento são um auxílio para as necessidades de busca por informação que uma pessoa pode ter \cite[p.~1]{Manning2008IIR} \cite[p.~1446]{Sanderson2012THIRR} \cite[p.~6]{Baeza-Yates1999}. 
    % OU  subáreas do conhecimento são uma ferramenta crucial na busca de informações.
    Esses sistemas de classificação facilitam a localização de informações específicas a partir de uma pergunta que uma pessoa pode fazer, mas é necessário saber como o sistema de classificação funciona para que o usuário possa tentar suprir sua necessidade, ou pelo menos será necessário um especialista no sistema (o bibliotecário) para lhe guiar.
    E mesmo assim, depois de adquirir os diversos materiais que podem responder à sua pergunta, esta pessoa ainda terá que conferir nos textos se estes contêm a informação desejada.
    
    % - RI em equipamentos eletromecânicos no início do século 20
    % -- O que fez surgir esses sistemas, a necessidade
    
    Os sistemas de classificação manual se mostram ineficazes devido ao surgimento crescente e constante de novas informações \cite[p.~6]{Baeza-Yates1999} desde o início do século 20. 
    Desta maneira, além da grande quantidade de novas pesquisas científicas sendo publicadas e livros surgindo, entre outros registros históricos, os quais precisam ser classificados, existe também o problema da classificação não poder abranger todo tipo de necessidade que tal material pode satisfazer \cite[p.~1444]{Sanderson2012THIRR}. 
    A preocupação com sistemas que possam indexar todo esse material e fornecer um acesso rápido às necessidades de informação de uma pessoa surgiu também no início do século 20 \cite{Bush:1979:WMT:1113634.1113638}, onde sistemas mecânicos de recuperação de informação foram vislumbrados.
    
    
    % - A área de pesquisa de RI que surgiu com a utilização de computadores para indexar informação
    % -- A utilização de computadores para fazer o serviço de RI
    
    A partir da criação de sistemas computacionais na década de 1940, foi vista a possibilidade de criação de sistemas que armazenassem informações e possibilitassem essa consulta rápida sobre as informações armazenadas, sendo necessário estabelecer algoritmos que retornassem informação relevante ao que o usuário do sistema procura. 
    Teve início nesse momento o campo científico da Recuperação de Informação (RI, do inglês \textit{Information Retrieval}) que envolve encontrar material (geralmente documentos) de natureza desestruturada (geralmente texto) que satisfaça uma necessidade de informação dentro de grandes acervos (geralmente armazenados em computadores) \cite[p.~1]{Manning2008IIR}.
    
    A lei de Moore diz que o crescimento da velocidade de processamento é contínuo \cite{Moore1975}, e de maneira similar existe uma duplicação constante da capacidade de armazenamento digital a cada dois anos \cite{Kryder2005}. 
    Logo, a necessidade de sistemas de Recuperação de Informação surge do crescimento exponencial das coleções de informação derivadas do crescimento de armazenamento, e consequente inabilidade das técnicas tradicionais de catalogação de lidar com isso \cite{Sanderson2012THIRR}.
    Ter um amontoado de conhecimento, informação, e não poder acessar o que é relevante de modo rápido não é interessante pois o desenvolvimento de pesquisas, por exemplo, fica comprometido e pode perder relevância \cite{Bush:1979:WMT:1113634.1113638}.
    
    % -- Conceitos fundamentais de RI
    A RI como uma disciplina de pesquisa iniciou no final da década de 50 a partir do uso de computadores na busca de referências de texto associadas com um assunto \cite[p.~3]{Sanderson2012THIRR}, as preocupações iniciais dessa área eram 
    \begin{enumerate*}[label=(\alph*)]
    \item \textit{como indexar documentos} e \item \textit{como recuperá-los},
    \end{enumerate*}
    sendo a busca da melhor maneira de executar tais tarefas o principal objetivo da RI.
    
    Logo no início do seu desenvolvimento as técnicas de RI buscaram se basear em sistemas existentes já consolidados no campo bibliotecário para indexar coleções de itens, tendo como uma técnica de abordagem clássica atribuir códigos numéricos a essas coleções, como por exemplo o feito pelo sistema de Classificação Decimal de Dewey \cite[p.~1446]{Sanderson2012THIRR}.
    No entanto, foi demonstrado por \citeonline{Cleverdon1959TESUIR} que um sistema baseado em palavras, como o sistema Uniterm proposto por \citeonline{Taube1952UTCI}, era tão bom e até melhor que outras abordagens clássicas, sendo a indexação por palavras posteriormente adotada pelos sistemas de RI \cite[p.~1446]{Sanderson2012THIRR}.
    
    Um exemplo do funcionamento de um sistema moderno de RI pode ser visto na Figura \ref{fig:diagrama-baeza2013-fig1-3}, onde está apresentado um fluxograma dos processos de indexação, recuperação, e ranqueamento de documentos. 
    Sendo o ranqueamento feito por sistemas de RI o interesse deste trabalho, alguns dos termos apresentados nessa figura serão abordados logo mais.
    
    \begin{figure}[H]
    \centering
    \includegraphics[width=1.0\textwidth]{img/baeza2013-figura-1-3.png}
    \caption{Processos de indexação, recuperação, e ranqueamento dos documentos, figura extraída de \citeonline[p.~8]{Baeza-Yates2011}.}
    \label{fig:diagrama-baeza2013-fig1-3}
\end{figure}
    
    Segundo \citeonline[p.~7]{Baeza-Yates2011} a arquitetura de funcionamento de um \textit{software} de RI inicia com o processo de indexação executado \textit{offline}, antes do sistema estar pronto para processar consultas, e a estrutura de indexação mais popular é a chamada de índice invertido.
    Após a indexação da coleção de documentos, o sistema está preparado para o processo de recuperação, no qual o usuário especifica sua consulta, que é modificada pelo sistema para ficar mais próxima do sistema de indexação utilizado.
    A consulta modificada é utilizada para obter o conjunto de documentos recuperados, os quais, idealmente, satisfazem as necessidades de informação do usuário.
    Por fim, os documentos recuperados são ranqueados pela sua pertinência de relevância para o usuário.
    
    A arquitetura apresentada na Figura \ref{fig:diagrama-baeza2013-fig1-3} inclui o ranqueamento feito pelos sistemas de RI mordernos, que são uma expansão dos sistemas de RI booleanos, estes últimos contam com uma indexação e recuperação mais simples que os sistemas de RI ranqueados como será esclarecido nas subseções a seguir.

    % -- Os primeiros sistemas de RI, booleanos 
\subsection{Métodos booleanos} \label{subsec:MétodosBooleanos}

    Uma consulta (chamada de \textit{query}) representa uma necessidade de informação a ser saciada por um sistema de RI, e essa consulta é composta de termos (um sinônimo para palavras) que nos primeiros desses sistemas era limitada a combinações lógicas e eram recuperados os documentos que tinham correspondência exata com ela \cite[p.~1446]{Sanderson2012THIRR}. 
    Este método de recuperação de informação é conhecido como recuperação booleana, e para indexar os documentos é utilizada, geralmente, uma matriz binária de incidência de termo-documento. 
    Exemplificamos uma matriz dessas na Tabela \ref{tab:matriz-incidência-termo-documento}, que é o exemplo dado por \citeonline[p.~3--4]{Manning2008IIR} de uma matriz de incidência de termo-documento para o livro \textit{Shakespeare’s Collected Works}, que reúne as obras completas de Shakespeare.

    \input{tex/table/matriz-incidencia-termo-documento.tex}
% \clearpage
    
    % -- A pesquisa de RI sendo desenvolvida, citar sistemas rankeados
    \subsection{Ranqueamento} \label{subsec:Ranqueamento}
Devido à limitação dos métodos booleanos de somente retornar resultados conforme a presença ou não dos termos da consulta nos documentos \cite[p.~100]{Manning2008IIR}, foi proposto em 1957 por Luhn e 1959 por Maron \textit{et al.} uma abordagem de recuperação ranqueada \cite[p.~1446]{Sanderson2012THIRR} a qual, em contraste com recuperação booleana, baseada nos termos de consulta estabelecia uma pontuação para cada artigo de modo probabilístico e retornava os artigos de modo ordenado e demonstraram que essa técnica sobressaía a recuperação booleana.

% Explicar o funcionamento de uma recuperação ranqueada, como é feito esse ranqueamento?
O procedimento fundamental para ranqueamento dos documentos, conforme os termos de consulta, consiste na atribuição de pontuação aos documentos a partir da contabilização do número de aparições (chama de frequência) de cada um dos termos no documento.
Essa pontuação é calculada considerando que além da frequência do termo, denotada como $\text{tf}_{\text{\textit{t},\textit{d}}}$ que é o número de ocorrências do termo \textit{t} em um documento \textit{d}, existe também a sua relevância, que depende do número de aparições do termo na coleção de documentos inteira.
Quanto mais um termo aparece na coleção menos relevante ele é, e este valor de relevância é denotado por $\text{idf}_{\text{\textit{t}}}$ que é o inverso da frequência de um termo \textit{t} em uma coleção de documentos.
Segundo \citeonline[p.~108]{Manning2008IIR} este valor da relevância é calculado do seguinte modo:

\begin{equation}
    \text{idf}_{\text{\textit{t}}} = \log{\frac{N}{\text{df}_{\text{\textit{t}}}}}.
\end{equation}

O valor resultante da relação entre a frequência do termo e o inverso da frequência nos documentos é chamado de $\text{tf-idf}_{\text{\textit{t},\textit{d}}}$ (\textit{term frequency-inverse document frequency}), sendo este valor um dos pesos mais utilizados para ranqueamento \cite[p.~107--110]{Manning2008IIR}, e é calculado  como segue:
\begin{equation}
    \text{tf-idf}_{\text{\textit{t},\textit{d}}}  = \text{tf}_{\text{\textit{t},\textit{d}}} \times \text{idf}_{\text{\textit{t}}}.
\end{equation}

\input{tex/table/tf-idf-exemplo.tex}

Na Tabela \ref{tab:exemplo-tf-idf} temos um exemplo de cálculo dos valores de tf-idf para posterior cálculo da pontuação para ranqueamento, conforme alguma determinada consulta. 
A pontuação de um documento \textit{d} é a soma dos pesos de tf-idf de cada termo \textit{t} em \textit{d}, sendo os termos \textit{t} presentes na consulta realizada \cite[p.~109]{Manning2008IIR}, representamos esse cálculo do seguinte modo:

\begin{equation}
    \label{eq:pontuação-simples-tf-idf}
    \text{Pontuação(\textit{q},\textit{d})} = \sum_{\textit{t} \in \textit{q}}^{} \text{tf-idf}_{\text{\textit{t},\textit{d}}}.
\end{equation}

% Quando uma consulta é feita são utilizados os valores 
% (onde Pontuação({\textit{auto},\textit{car}},DocX))  

Utilizando a Equação \ref{eq:pontuação-simples-tf-idf} uma consulta com os termos \textit{auto car} retornaria no seu ranqueamento os documentos com a seguinte pontuação, calculamos Pontuação(\{\textit{auto},\textit{car}\},DocX) para cada documento, por exemplo:
\begin{itemize}
    \setlength\itemsep{-0.2em}
    \item Doc1: 50,79
    \item Doc2: 75,24
    \item Doc3: 39,60
\end{itemize}

A ordenação dos documentos apresentados como resultado à consulta \textit{auto car} seria então a seguinte: 1\textordmasculine{} - Doc2; 2\textordmasculine{} - Doc1; e 3\textordmasculine{} - Doc3, que se observamos a Tabela \ref{tab:exemplo-tf-idf} é um bom resulto já que o Doc2 contém uma grande frequência do termo \textit{auto} e o Doc3 não possui este termo.


Ao longo dos anos foi demonstrada a superioridade da recuperação ranqueada sobre a recuperação booleana \cite{Jones:1981:IRE:539571}, e são as técnicas de recuperação ranqueadas que trazem maior interesse para a área de Mineração de Textos, em específico estamos interessados nos modelos vetoriais e os modelos probabilísticos de RI que são evoluções da recuperação ranqueada.
% ESTOU PENSANDO EM JÁ CITAR O BM25 no parágrafo acima.
% -- Em algum ponto mencionar que Recuperação de Informação (Information Retrieval) não deve ser confundido com Procura de Informação (Information Search), pois a Procura de Informação é o campo que estuda a interação das pessoas com sistemas de recuperação de informação.




    \newpage

    \section{Mineração de Texto} \label{sec:MineraçãoTexto}
% Introdução a Mineração de Texto
% - Falar de IA?
% -- Surgimento do processo de KDD
% -- É um subramo do KDD, KDT
A Mineração de Textos (MT) é definida como o processo de extrair conhecimento implícito de dados textuais \cite{Jo2018TMCIBDC,Feldman:2006:TMH:1076381} e por isso é às vezes tratada como \textit{knowledge discovery in text} (livremente traduzido para descoberta de conhecimento em texto) \cite{Kodratoff:1999:KDT:646358.689959, Feldman:1995:KDT:3001335.3001354}, sendo análogo ao termo \textit{knowledge discovery in data} (KDD) que se refere à Mineração de Dados, ramo da Inteligência Artificial que dá suporte a MT. 
Apesar de haver um uso sinônimo entre Mineração de Dados e KDD, alguns autores tratam a Mineração de Dados como somente uma parte desse processo de descoberta de conhecimento \cite[p.~6]{Han:2011:DMC:1972541}, sendo este um processo iterativo composto pelas seguintes fases (ou etapas) segundo \citeonline[p.~6-7]{Han:2011:DMC:1972541}:
% - Passos da MT
% 1. Data cleaning (to remove noise and inconsistent data)
% 2. Data integration (where multiple data sources may be combined)3
% 3. Data selection (where data relevant to the analysis task are retrieved from the  database)
% 4. Data transformation (where data are transformed and consolidated into forms appropriate for mining by performing summary or aggregation operations)4
% 5. Data mining (an essential process where intelligent methods are applied to extract data patterns)
% 6. Pattern evaluation (to identify the truly interesting patterns representing knowledge  based on interestingness measures—see Section 1.4.6)
% 7. Knowledge presentation (where visualization and knowledge representation techniques are used to present mined knowledge to users)
\begin{enumerate}
    \item \textbf{Limpeza dos dados}: remoção de ruído e dados inconsistentes;
    \item \textbf{Integração dos dados}: combinação de múltiplas fontes de dados;
    \item \textbf{Seleção dos dados}: dados relevantes para a tarefa de análise são recuperados do banco de dados;
    \item \textbf{Transformação dos dados}: dados são transformados e consolidados em formas apropriadas para mineração sendo realizadas, por exemplo, ações de agregação ou resumo;
    \item \textbf{Mineração dos dados}: métodos inteligentes são aplicados para extrair padrões de dados;
    \item \textbf{Avaliação de padrões}: são identificados os padrões que realmente tão interessantes para representar o conhecimento baseado em medidas de nível de interesse;
    \item \textbf{Apresentação do conhecimento}: o conhecimento minerado é apresando aos usuários por meio de técnicas de visualização e representação de conhecimento.
\end{enumerate}
% Certainly, text mining derives much of its inspiration and direction from seminal research on data mining. Therefore, it is not surprising to find that text mining and data mining systems evince many high-level architectural similarities. 

É importante notar essas 7 etapas de desenvolvimento de Mineração de Dados para abordamos a definição de MT pois esta deriva muitas técnicas desenvolvidas na pesquisa do campo de Mineração de Dados para seu campo de aplicação, logo sistemas baseados em ambas áreas vão apresentar similaridades arquiteturais \cite{Feldman:2006:TMH:1076381}. 

% Text mining can be broadly defined as a knowledge-intensive process in which a user interacts with a document collection over time by using a suite of analysis tools. In a manner analogous to data mining, text mining seeks to extract useful information from data sources through the identification and exploration of interesting patterns.

% Because data mining assumes that data have already been stored in a structured format, much of its preprocessing focus falls on two critical tasks: Scrubbing and normalizing data and creating extensive numbers of table joins. In contrast, for text mining systems, preprocessing operations center on the identification and extraction of representative features for natural language documents. These preprocessing operations are responsible for transforming unstructured data stored in document collections into a more explicitly structured intermediate format, which is a concern that is not relevant for most data mining systems.

A Mineração de Dados assume que os dados, que vão ser tratados durante seu processo, já foram armazenados em um formato estruturado, logo a maior parte de seu pré-processamento vai estar ligado às etapas 1 e 2 do processo de KDD citado, as de limpeza e integração dos dados \cite{Feldman:2006:TMH:1076381}. % Talvez seja necessário ao falar de linguagem natural dar um exemplo e um contra exemplo? Português e Java?
Já na MT, como os dados de trabalho são textos, sendo texto configurado como dados desestruturados que consistem de \textit{strings} (palavras) organizadas de forma coerente e sendo pertencentes a uma linguagem natural \cite{Jo2018TMCIBDC}, temos que as operações de pré-processamento vão estar mais focadas em etapas adicionais, prévias às citadas para o processo de KDD, sendo estas novas direcionadas à identificação e extração de \textit{features} (atributos) representativas para documentos escritos em linguagem natural, transformando os dados não estruturados, que estão armazenados em coleções de documentos, em um formato mais explicitamente estruturado \cite{Feldman:2006:TMH:1076381}.
% Text is defined as the unstructured data which consists of strings which are called words [82]

% Na década de 70 houve o surgimento de diversas técnicas de gerenciamento de banco de dados, como por exemplo a 

% -- Utiliza de várias áreas, falar delas
% -- Utiliza das técnicas de RI para indexar os textos em algumas de suas aplicações
% Text mining preprocessing operations include a variety of different types of techniques culled and adapted from information retrieval, information extraction, and computational linguistics research that transform raw, unstructured, original-format content (like that which can be downloaded from PubMed) into a carefully structured, intermediate data format. Knowledge discovery operations, in turn, are operated against this specially structured intermediate representation of the original document collection.

As operações de pré-processamento para MT utilizam de várias técnicas adaptadas dos campos de Recuperação de Informação, extração de informação e linguística computacional para transformar as coleções de documentos desestruturados em dados intermediários cuidadosamente estruturados \cite[p.2-3]{Feldman:2006:TMH:1076381}. 
Essa estrutura intermediária é definida por um modelo representacional dos documentos de texto composto por um conjunto de \textit{features}, sendo sempre preferidos os modelos com menor número de variáveis significativas para a representação.

% Pensei em aqui contextualizar um pouco sobre os diversos campos que dão suporte ao KDD (Mineração de Dados) e à MT e colocar uma imagem parecida com a do \cite{Han:2011:DMC:1972541} na página 23 para mostrar os campos que dão suporte a MT

A definição das \textit{features} para MT busca tirar proveito dos mais variados elementos presentes em um documento escrito em linguagem natural, no entanto é necessário um cuidado pois existe um grande número de palavras, frases e outros artefatos que podem comprometer o desempenho de um Sistema de Mineração de Texto (SMT) ou tornar a tarefa infactível %(NECESSITA DE FONTES?)
, por isso a necessidade de identificar as melhores variáveis, que trazem mais informação sobre o texto. 
Nesse ponto que a MT se auxilia de técnicas de RI para incrementar seu grupo de variáveis, sendo alguns, como por exemplo o BM25 para RI ranqueada, utilizadas em competições de determinação de perfil de autores (CITAR Weren2014EMFAP e procurar outros).

% -- Já devo diferenciar aqui
		% ísticos que baseiam-se na %--------------------------------------------------------------------------------------
% Este arquivo contém a sua metodologia
%--------------------------------------------------------------------------------------
\chapter{Materiais e Métodos} \label{ch:MateriaisMétodos} %Uma label é como você referencia uma seção no texto com a tag \ref{}
    Neste capítulo será apresentado como será feita a avaliação do desempenho das funções de ranqueamento na Mineração de Texto.
    
    Para utilização da função de ranqueamento BM25 como variável em tarefas de mineração de texto é necessário efetuar o armazenamento e a indexação dos textos do conjunto de treinamento, e para esta tarefa serão utilizadas ferramentas de armazenamento em banco de dados (não relacionais?) que fornecem implementações do BM25 para consulta, estas serão apresentadas na Seção \ref{sec:Armazenamento-e-indexação}.
    
    Se faz necessário também definir métricas para avaliação do ganho de desempenho com as variáveis de RI, assim como elencar as bases de dados que vão servir para efetuar essa avaliação, disposto logo mais nas Seções \ref{sec:Métricas-de-desempenho-para-avaliação} e \ref{sec:Bancos-de-dados-para-teste}.

\section{Armazenamento e indexação} \label{sec:Armazenamento-e-indexação}

    Para armazenar e indexar os bancos de dados das tarefas de Mineração de Textos, e possibilitar assim o cálculo da função BM25 para cada exemplo, serão utilizadas as seguintes tecnologias que fazem implementações do BM25:
    \begin{itemize}
        \item Elasticsearch % https://www.elastic.co/guide/en/elasticsearch/reference/current/index-modules-similarity.html
        % https://www.elastic.co/pt/blog/practical-bm25-part-2-the-bm25-algorithm-and-its-variables
        \item ArangoDB
        % https://www.arangodb.com/docs/stable/aql/views-arango-search.html#bm25
        % https://www.arangodb.com/news/introducing-arangodb-3-4/
        \item Zettair
        % http://www.seg.rmit.edu.au/zettair/doc/Readme.html
        \item Apache Spark
    \end{itemize}
    % Links dos sites das ferramentas como notas de rodapé.
    % Citar a versão de cada ferramenta que vai ser utilizada.
    % 
    
\section{Bancos de dados para teste}  \label{sec:Bancos-de-dados-para-teste}
    Os banco de dados de teste selecionados para avaliação são os utilizados na PanCLEF 2019 % necessário citar a PanCLEF anteriormente

    \begin{itemize}
        \item Bots and Gender Profiling PAN @ CLEF 2019
        % https://pan.webis.de/clef19/pan19-web/author-profiling.html
        \item Hyperpartisan News Detection - PAN @ SemEval 2019
        %  https://pan.webis.de/semeval19/semeval19-web/index.html
    \end{itemize}
    % Descreuioes a partir da tarefa no site
    % Definir as soluçcoes disponiveis que vou utilizar (Nao pode ter RI)
 
% Metodologia:

% Vão ser elencadas as tecnologias para armazenamento e indexação dos dados por meio do BM25:
% * Elasticsearch
% * APache Spark
% * ArangoDB
% * Zettair

% Os bancos de teste serão:
% * PanCLEF Hyperpartisan 2019
% https://pan.webis.de/semeval19/semeval19-web/leaderboard.html
% * A definir
% * Bots and Gender Profilin % PAN @ CLEF 2019 

\section{Variáveis de RI sugeridas}  \label{sec:Variáveis-de-RI-sugeridas}
    
%  Por exemplo, para a classificação binária do PANCLEF 2019 Hyperpartisan podem ser criadas as variáveis a seguir:
% * avg_0 
% * count_0
% * sum_0
% * avg_1
% * count_1
% * sum_1

% baseadas no weren (2014)


\section{Métricas para avaliação de desempenho}  \label{sec:Métricas-para-avaliação-de-desempenho}
    Será avaliado o desempenho das ferramentas utilizadas para indexação e consulta, por meio da avaliação temporal do desempenho computacional, e também o ganho de desempenho propiciado pelo uso das variáveis de RI em termos das métricas de classificadores de Mineração de Texto. % Essas métricas tem que ser citadas no referencial teórico.
    
    \subsection{Desempenho computacional das ferramentas}  \label{sec:Desempenho-computacional}
        Para avaliar o desempenho das ferramentas de armazenamento e indexação utilizaremos duas métricas temporais:
        \begin{itemize}
            \item \textbf{TIME\_INDEX}: Tempo de execução para indexar o conjunto de treinamento de cada um dos banco de dados de teste elencados na Seção \ref{sec:Bancos-de-dados-para-teste}. Dadas as quatro diferentes ferramentas de indexação e os dois banco de dados selecionados, serão computadas 8 TIME\_INDEX para comparação;
            
            \item \textbf{TIME\_QUERY}: Tempo para consulta de cada exemplo do conjunto de teste e geração das variáveis sugeridas na Seção \ref{sec:Variáveis-de-RI-sugeridas} para o item específico. Dadas as 12 variáveis sugeridas para criação distribuídas nos dois banco de dados selecionados, e dadas as 4 ferramentas de indexação, 48 TIME\_QUERY serão computadas.  
        \end{itemize}
        
        Essas variáveis serão computadas para cada uma das 4 ferramentas selecionadas sendo executadas no mesmo sistema computacional a fim de oferecer maior confiança aos números obtidos. 
        O sistema computacional a ser utilizado para efetuar o experimento ainda não está definido, mas será especificado nos resultados.
    
% O teste de desempenho será feito comparando as ferramentas de indexação, tempo para indexar o treino de cada banco (TIME_TRN) em cada ferramenta.

% O tempo para consultar para cada linha do teste as variáveis a serem criadas (no caso de classificação binária iremos definir agregações do tipo count, sum e avg). Por exemplo, para a classificação binária do PANCLEF 2019 Hyperpartisan podem ser criadas as variáveis a seguir:
% * avg_0 
% * count_0
% * sum_0
% * avg_1
% * count_1
% * sum_1
    \subsection{Desempenho de classificador}  \label{sec:Desempenho-de-classificador}
    O ganho de desempenho propiciado pelas variáveis de RI criadas será mensurado por métricas de desempenho de classificadores da Mineração de Dados, as mesmas da Mineração de Texto.
    Nesse estudo serão computadas as seguintes métricas:
    \begin{itemize}
        \item \textbf{CLF\_ACC}: Acurácia do classificador.
        \item \textbf{CLF\_AUC}: Área sobre a curva ROC.
        \item \textbf{CLF\_F1}: F1-Score
        \item \textbf{CLF\_PRE}: Precisão do classificador.
        \item \textbf{CLF\_REC}: Revocação do classificador.
    \end{itemize}
% Ao final será comparado o ganho de desempenho (acurácia, precisão, recall e F1-score) nas melhores soluções dos bancos definidos a partir da adição das variáveis criadas.
% Por exemplo para a PANCLEF 2019 Hyperpartisan será utilizada a solução que está disponível no github com melhor score e ela será executada novamente para conferir o resultado obtido de acurária que eles dizem e então será rodado novamente com a adição das variáveis de RI (BM25).


ADICIONAR IMAGEM DE VISÃO GERAL DA METODOLOGIA


% \subsection{Subseção de exemplo 1 - Referenciando seções} \label{subsec:subsec1}






%--------------------------------------------------------------------------------------
% Insere a seção de cronograma
% Está comentada porque só é necessária no TCC I
%--------------------------------------------------------------------------------------
% %--------------------------------------------------------------------------------------
% Insere a seção de cronograma
%--------------------------------------------------------------------------------------

\section{Cronograma} \label{sec:Cronograma}

A Tabela \ref{tab:cronograma} mostra o cronograma de atividades a serem executadas para o Trabalho de Conclusão II (TCC II), com base no calendário do período 2019.2 da UNIVASF, definido pelo Calendário Acadêmico 2019 da instituição.

\begin{table}[!thb]
	%\huge
    \centering
    \caption{Cronograma das atividades previstas para o TCC II.}
    \begin{adjustbox}{max width=\textwidth}
    \begin{tabular}{p{6.5cm}|c|c|c|c|c|c}
        \toprule
        \textbf{Atividade}
        & Set & Out & Nov & Dez & Jan & Fev
        \\ \hline
        Definição e obtenção dos corpus para avaliação 
        & X   &     &     &     &     &          
        \\ \hline
        Inspeção e seleção das soluções com código fonte disponível
        & X   &     &     &     &     &          
        \\ \hline
        Instalação e familiarização com as ferramentas de arquivamento e indexação
        & X   & X   &     &     &     &          
        \\ \hline
        Indexação do conjunto de treinamento dos corpus
        &     & X   & X   &     &     &          
        \\ \hline
        Adição dos atributos de RI às soluções selecionadas
        &     & X   & X   &     &     &    
        \\ \hline
        Mineração dos dados por meio da reprodução das soluções selecionadas com/sem adição dos atributos de RI
        &     &     & X   & X   & X   &  
        \\ \hline
        Escrita do TCC II                       
        & X   & X   & X   & X   & X   &         
        \\ \hline
        Defesa do TCC II                        
        &     &     &     &     & X   &        
        \\
        \bottomrule
    \end{tabular}
    \end{adjustbox}
    
    \label{tab:cronograma} 
    % \legend{\textbf{Fonte:} O autor.}
\end{table}

		%--------------------------------------------------------------------------------------
% Insere a seção de cronograma
%--------------------------------------------------------------------------------------

\section{Cronograma} \label{sec:Cronograma}

A Tabela \ref{tab:cronograma} mostra o cronograma de atividades a serem executadas para o Trabalho de Conclusão II (TCC II), com base no calendário do período 2019.2 da UNIVASF, definido pelo Calendário Acadêmico 2019 da instituição.

\begin{table}[!thb]
	%\huge
    \centering
    \caption{Cronograma das atividades previstas para o TCC II.}
    \begin{adjustbox}{max width=\textwidth}
    \begin{tabular}{p{6.5cm}|c|c|c|c|c|c}
        \toprule
        \textbf{Atividade}
        & Set & Out & Nov & Dez & Jan & Fev
        \\ \hline
        Definição e obtenção dos corpus para avaliação 
        & X   &     &     &     &     &          
        \\ \hline
        Inspeção e seleção das soluções com código fonte disponível
        & X   &     &     &     &     &          
        \\ \hline
        Instalação e familiarização com as ferramentas de arquivamento e indexação
        & X   & X   &     &     &     &          
        \\ \hline
        Indexação do conjunto de treinamento dos corpus
        &     & X   & X   &     &     &          
        \\ \hline
        Adição dos atributos de RI às soluções selecionadas
        &     & X   & X   &     &     &    
        \\ \hline
        Mineração dos dados por meio da reprodução das soluções selecionadas com/sem adição dos atributos de RI
        &     &     & X   & X   & X   &  
        \\ \hline
        Escrita do TCC II                       
        & X   & X   & X   & X   & X   &         
        \\ \hline
        Defesa do TCC II                        
        &     &     &     &     & X   &        
        \\
        \bottomrule
    \end{tabular}
    \end{adjustbox}
    
    \label{tab:cronograma} 
    % \legend{\textbf{Fonte:} O autor.}
\end{table}
% 		\chapter{Resultados} \label{ch:Resultados}
%  Os resultados seguem a metodologia

% Definir qual o top_k utilizado
% Verificar em Were2014 novamente
% Padrão do zettair (se não me engano é 20)
% Artigo de fulano não diz

% Capítulo de Resultados 

% -> Introdução: "Este capítulo apresenta os resultados obtidos e discute-os..." 
Este capítula apresenta os resultados obtidos neste estudo investigativo do desempenho de atributo de RI em classificadores de Mineração de Texto.

Nas subseções a seguir primeiro é abordada a configuração experimental utilizada para realizar o estudo, e logo em seguida é apresentada uma visão geral das soluções selecionadas dos corpus DB\_AUTHORPROF e DB\_HYPERPARTISAN, onde os pré-processamento realizado e os classificadores utilizados em cada uma das soluções são descritos brevemente.
Por fim, são apresentados os resultados mensurados por meio das medidas escolhidas para avaliação de desempenho, na subseção X estão as medidas de desempenho das ferramentas de armazenamento e indexação e na subseção são expostas as medidas de desempenho dos classificadores.

% Na subseção a seguir são abordados os resultados referentes às ferramentas de indexação.
% Na subseção posterior são abordados os resultados referente aos desempenho das variáveis de RI em classificadores.

\section{Configuração experimental} \label{sec:resex1}
% 4.0 Setup experimental 

	Para programação e execução dos experimentos foi utilizado o sistema computacional disponível para o autor, com a configuração disposta na Tabela 4.1.

	As ferramentas de armazenamento e indexação receberam os mesmos parâmetros de refinamento na configuração de suas funções BM25, sendo estes configurados em $k_1 = 1.2$, $k_3 = 0$ e $b = 0.75$.
	O parâmetro $k_3$, para escalonar a frequência de termos na consulda, é exclusivo para o Zettair, as demais ferramentas não implementam este parâmetro.

	% Fixação do número aleatório do Python e das bibliotecas utilizadas nas soluções, quando isto não era feito originalmente, para permitir reproducibilidade exata dos resultados obtidos. 

\section{Visão geral das soluções selecionadas} \label{sec:resex1}
	% Falar brevemento sobre o pré-processamento;
	% Falar sobre o classificador utilizado; e
	% Indicar o Notebook de cada solução para mais detalhes
	\subsection{Soluções para o corpus DB\_HYPERPARTISAN}

	\subsection{Soluções para o corpus DB\_AUTHORPROF}

\section{Desempenho das ferramentas de armazenamento e indexação} \label{sec:resex1}
	Para cálculo das variáveis TIME_INDEX e TIME_QUERY, conforme sugeridas no Capítulo 3, foi utilizada a linguagem de programação Python na qual foi implementada uma classe abstraindo a indexação e cálculo das variáveis de RI nas 3 ferramentas de Indexação, chamada IndexToolManager. 

	Esta classe foi central para todo o estudo. 

	\subsection{Tempo de indexação}
		Para cálculo das variáveis TIME_INDEX foi criado um script python nomeado time_index.py, o qual utilizou da classe IndexToolManager em duas funções feitas para executar a indexação dos banco de dados, DB_AUTHORPROF e DB_HYPERPARTISAN, nas 3 ferramentas, ARANGO, ELASTIC e ZETTAIR. 

		

		Como as ferramentas ARANGO e ELASTIC se assemelham bastante a sistemas gerenciadores de bancos de dados, preparados, inclusive, para distribuição geográfica dos dados, há de ser citada essa grande diferença deles para o ZETTAIR, este último que é somente um sistema para Indexação em lotes e consulta local dos dados, não permitindo, por exemplo, adição de novos documentos em um índice. A ação de inserção unitária é um procedimento comum em SGBDs. 

		

		A operação de indexação foi executada de dois modos, em lote e unitária, sendo que o ZETTAIR só permite a inserção em lote. 

		

		Na Figura 4.1 podem ser vistos os resultados do tempo para inserção em lote dos documentos dos corpus  em cada ferramenta. 

		

		O ZETTAIR é a ferramenta mais rápida para completar a indexação de ambos os corpus selecionados, levando somente 2,x segundos para indexar os 300 mil documentos do corpus DB_AUTHORPROF. 

		Dentre os SGBDs avaliados, vemos que o ELASTIC tem  melhor desempenho que o ARANGO para inserções em lote, gastando 15 segundos para indexar os 300 mil documentos. 

		

		Para operação de inserção unitária foi levado em conta o tempo total para indexação dos 300 mil documentos, inseridos em sequência, na Figura 4.2 estão dispostos os tempos totais para indexação de todos os documentos dos corpus com ambas ferramentas. 

	\subsection{Tempo de consulta}

\section{Desempenho dos classificadores com atributos de RI} \label{sec:resex1}

	\subsection{DB\_HYPERPARTISAN}

	\subsection{DB\_AUTHORPROF}

% Fluxograma das alterações feitas nos códigos com exemplos de trechos alterados
% Rosalvo disse que é para colocar no Apêndice

% Zettair, colocar detalhe dos modos de operação para consulta, iterativa e consulta única

% ganho de informação das 6 variáveis nas soluções

% Citar as técnicas só de um, descrições, citar algum livro ou página

% Criar rede neural para comparar à solução do DB_AUTHORPROF à parte 


		%--------------------------------------------------------------------------------------
% Este arquivo contém a sua conclusão
%--------------------------------------------------------------------------------------
\chapter{Considerações Finais e Trabalhos Futuros} \label{ch:ConsideraçõesFinais}

% recapitular tudo, qual o propósito
% 1 parágrafo

% falar o que vi de fato nos resultados
% -- foi visto que para a base X, Y aconteceu provavelmente  devido a Z
 
% reforçar as melhores ferramentas, porque são resultados de desempenho mais claros

% limitações do que foi feito

% 
\section{Trabalhos futuros}

% \lipsum[55]

	\postextual
		\bibliography{tex/references.bib}
		\input{tex/anexos}

\end{document}
